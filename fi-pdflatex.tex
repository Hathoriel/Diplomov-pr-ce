%%%%%%%%%%%%%%%%%%%%%%%%%%%%%%%%%%%%%%%%%%%%%%%%%%%%%%%%%%%%%%%%%%%%
%% I, the copyright holder of this work, release this work into the
%% public domain. This applies worldwide. In some countries this may
%% not be legally possible; if so: I grant anyone the right to use
%% this work for any purpose, without any conditions, unless such
%% conditions are required by law.
%%%%%%%%%%%%%%%%%%%%%%%%%%%%%%%%%%%%%%%%%%%%%%%%%%%%%%%%%%%%%%%%%%%%

\documentclass[
  printed, %% This option enables the default options for the
           %% digital version of a document. Replace with `printed`
           %% to enable the default options for the printed version
           %% of a document.
  twoside, %% This option enables double-sided typesetting. Use at
           %% least 120 g/m² paper to prevent show-through. Replace
           %% with `oneside` to use one-sided typesetting; use only
           %% if you don’t have access to a double-sided printer,
           %% or if one-sided typesetting is a formal requirement
           %% at your faculty.
  table,   %% This option causes the coloring of tables. Replace
           %% with `notable` to restore plain LaTeX tables.
  nolof,     %% This option prints the List of Figures. Replace with
           %% `nolof` to hide the List of Figures.
  nolot,     %% This option prints the List of Tables. Replace with
           %% `nolot` to hide the List of Tables.
  %% More options are listed in the user guide at
  %% <http://mirrors.ctan.org/macros/latex/contrib/fithesis/guide/mu/fi.pdf>.
]{fithesis3}
%% The following section sets up the locales used in the thesis.
\usepackage[resetfonts]{cmap} %% We need to load the T2A font encoding
\usepackage[T1,T2A]{fontenc}  %% to use the Cyrillic fonts with Russian texts.
\usepackage[
  main=czech, %% By using `czech` or `slovak` as the main locale
                %% instead of `english`, you can typeset the thesis
                %% in either Czech or Slovak, respectively.
  english, german, russian, czech, slovak %% The additional keys allow
]{babel}        %% foreign texts to be typeset as follows:
%%
%%   \begin{otherlanguage}{german}  ... \end{otherlanguage}
%%   \begin{otherlanguage}{russian} ... \end{otherlanguage}
%%   \begin{otherlanguage}{czech}   ... \end{otherlanguage}
%%   \begin{otherlanguage}{slovak}  ... \end{otherlanguage}
%%
%% For non-Latin scripts, it may be necessary to load additional
%% fonts:
\usepackage{paratype}
\def\textrussian#1{{\usefont{T2A}{PTSerif-TLF}{m}{rm}#1}}
%%
%% The following section sets up the metadata of the thesis.
\thesissetup{
    date          = \the\year/\the\month/\the\day,
    university    = mu,
    faculty       = fi,
    type          = mgr,
    author        = Bc. Lukáš Kotol,
    gender        = m,
    advisor       = {RNDr. Miloš Liška, Ph.D.},
    title         = {Autentizační a autorizační infrastruktura pro videokonferenční prostředí},
    TeXtitle      = {Autentizační a autorizační infrastruktura pro videokonferenční prostředí},
    keywords      = {OpenID Connect, autentizace, autorizace, rezervační systém Shongo, Adobe Connect, CESNET, videokonference, webkonference},
    TeXkeywords   = {OpenID Connect, autentizace, autorizace, rezervační systém Shongo, Adobe Connect, CESNET, videokonference, webkonference},
    abstract      = {Diplomová práce se zabývá návrhem a implementací nové autentizační a autorizační infrastruktury pro webkonferenční prostředí provozované sdružením CESNET, které sestavá ze služeb meetings.cesnet.cz a Adobe Connect. Implementace nové autentizační a autorizační infrastruktury je založena na technologii OpenID Connect a využívá Proxy IdP v infrastruktuře CESNETu. Součástí práce je také analýza stávající autentizační a autorizační infrastruktury, na kterou implementace navazuje. \par První část práce se věnuje analýze požadavků a návrhu nové autentizační a autorizační infrastruktury. Druhá část představuje použité technologie, které byly využity při implementaci. Třetí a zároveň poslední část práce popisuje samotnou implementaci nové autentizační a autorizační vrstvy pro služby meetings.cesnet.cz a Adobe Connect},
    thanks        = {These are the acknowledgements for my thesis, which can

                     span multiple paragraphs.},
    bib           = example.bib,
}
\usepackage{makeidx}      %% The `makeidx` package contains
\makeindex                %% helper commands for index typesetting.
%% These additional packages are used within the document:
\usepackage{paralist} %% Compact list environments
\usepackage{amsmath}  %% Mathematics
\usepackage{amsthm}
\usepackage{amsfonts}
\usepackage{url}      %% Hyperlinks
\usepackage{markdown} %% Lightweight markup
\usepackage{listings} %% Source code highlighting
\usepackage{hyperref}
\usepackage{xcolor}
\usepackage{bera}

\colorlet{punct}{red!60!black}
\definecolor{background}{HTML}{EEEEEE}
\definecolor{delim}{RGB}{20,105,176}
\colorlet{numb}{magenta!60!black}

\lstdefinelanguage{json}{
    basicstyle=\footnotesize\ttfamily,
    showstringspaces=false,
    breaklines=true,
    frame=lines,
    backgroundcolor=\color{background},
    literate=
     *{0}{{{\color{numb}0}}}{1}
      {1}{{{\color{numb}1}}}{1}
      {2}{{{\color{numb}2}}}{1}
      {3}{{{\color{numb}3}}}{1}
      {4}{{{\color{numb}4}}}{1}
      {5}{{{\color{numb}5}}}{1}
      {6}{{{\color{numb}6}}}{1}
      {7}{{{\color{numb}7}}}{1}
      {8}{{{\color{numb}8}}}{1}
      {9}{{{\color{numb}9}}}{1}
      {:}{{{\color{punct}{:}}}}{1}
      {,}{{{\color{punct}{,}}}}{1}
      {\{}{{{\color{delim}{\{}}}}{1}
      {\}}{{{\color{delim}{\}}}}}{1}
      {[}{{{\color{delim}{[}}}}{1}
      {]}{{{\color{delim}{]}}}}{1},
}


\lstset{
  basicstyle      = \footnotesize\ttfamily,%
  identifierstyle = \color{black},%
  keywordstyle    = \color{blue},%
  keywordstyle    = {[2]\color{cyan}},%
  keywordstyle    = {[3]\color{olive}},%
  stringstyle     = \color{teal},%
  commentstyle    = \itshape\color{magenta},
  language=XML,
  morekeywords={Location},
  backgroundcolor=\color{background},
  frame=lines
}
\usepackage{floatrow} %% Putting captions above tables
\floatsetup[table]{capposition=top}
\begin{document}

\chapter{Úvod}
Komunikace prostřednictvím videopřenosů a multimédií hraje v současném světě informačních technologií důležitou roli. Bez těchto moderních komunikačních prostředků si v dnešní době jen těžko dovedeme představit spolupráci v kolektivu, který tvoří osoby nacházející se na různých místech naší planety. Klíčovou vlastností zmíněných videokonferenčních systémů je dostupnost napříč uživatelskými organizacemi, jejichž spolupráci mají usnadňovat. Aby bylo možné propojení rozdílných organizací, je důležité mít videokonferenční platformu s vybudovanou autentizační a autorizační infrastrukturou, která tyto organizace dokáže integrovat. Pokud je autentizační infrastruktura navíc sdílena mezi další poskytovatele služeb, benefity představované platformy organizací a provozovatelů služeb, tzv. federace identit jsou nemalé. Jedná se například o fakt, že poskytovatelé služeb jsou osvobozeni od správy uživatelských údajů a hesel. Správce služby také definuje, jaký typ uživatele smí k jeho službě přistupovat, bez toho aby tyto politiky musel implementovat. Další výhodou federace identit je skutečnost, že uživateli stačí jen jedno přihlašovací jméno a heslo k přístupu ke všem poskytovaným službám. Uživatel přihlašovací údaje navíc vždy zadává na důvěrně známé webové stránce své organizace. V neposlední řadě, v případě kompromitace  přihlašovacích údajů si uživatel může v jednom kroku vygenerovat nové pro celou federaci identit. 

Cílem této diplomové práce je navrhnout a implementovat novou autentizační a autorizační infrastrukturu pro současné video a webkonferenční prostředí, které spravuje sdružení CESNET\footnote{CESNET, \url{www.cesnet.cz}}. Vytvořená autentizační a autorizační infrastruktura je založena na technologii OpenID Connect\footnote{OpenID Connect, \url{www.openid.net/connect/}}  a je integrována do stávající Proxy IdP v infrastruktuře CESNETu. Na novou autentizační a autorizační infrastrukturu jsou navázány služby \texttt{meetings.cesnet.cz} \cite{shongoapi} a Adobe Connect\footnote{Adobe Connect, \url{www.adobe.com/products/adobeconnect.html}}, které také provozuje sdružení CESNET. Zmíněné videokonferenční služby mohou využívat uživatelé z organizací, které patří do České akademické federace identit eduID.cz \footnote{eduID.cz, \url{www.eduid.cz}} a její mezinárodní obdoby eduGAIN \footnote{eduGAIN, \url{www.edugain.org}}. Služba meetings.cesnet.cz slouží uživatelům k rezervaci výpočetních zdrojů a virtuálních místností, v kterých se uskutečňují následné videokonferenční spojení. Komerční platforma Adobe Connect, provozovaná na URL connect.cesnet.cz\footnote{Adobe Connect CESNET Webkonference, \url{www.connect.cesnet.cz}}, umožňuje autentizovaným uživatelům uskutečňovat videokonferenční a webkonferenční schůzky v předem rezervovaných virtuálních místnostech. Zmíněná platforma Adobe Connect navíc podporuje textovou diskuzi, sdílení obsahu dokumentů, kreslící tabule a pracovní plochy počítače v prostředí prohlížeče. 

Výsledkem mé práce je plně funkční autentizační a autorizační infrastruktura zprovozněná v produkčním prostředí webkonferenční platformy sdružení CESNET. Vytvořenou autentizační a autorizační infrastrukturu jsem navrhl a implementoval na serveru \texttt{shongo-auth.cesnet.cz} pomocí technologie OpenID Connect. Implementovaná infrastruktura je integrována do Proxy IdP provozované sdružením CESNET tak, aby autentizovala a autorizovala uživatele, využívající služeb portálů meetings.cesnet.cz a connect.cesnet.cz. Nová implementace oproti původní rozšiřuje způsoby přihlášení o možnosti použít účty společností Google, Facebook a dalších mezinárodních akademických organizací integrovaných do federace identit eduGAIN. Další předností nové implementace je vynechání autorizačního procesu pomocí externí webové služby Perun\footnote{Perun, \url{www.perun-aai.org}}, která poskytovala základní údaje o autentizovaném uživateli. Nyní jsou údaje přihlášeného uživatele zprostředkovány v rámci autentizačního procesu díky technologii OpenID Connect. Novou implementací došlo tedy k významnému zjednodušení složitosti autentizační a autorizační infrastruktury. \par

Text této diplomové práce je členěn následujícím způsobem. První část se věnuje úvodu. Po úvodu následuje druhá kapitola, ve které popisuji návrh infrastruktury autorizačního a autentizačního systému. V třetí kapitole se věnuji použitým technologiím, které jsem využil v rámci implementace diplomové práce. V následující čtvrté kapitole se zabývám popisem implementace. Poslední pátá kapitola shrnuje výsledky mé práce v závěru. 

\chapter{Návrh autorizační a autentizační infrastruktury}
V této kapitole se budu věnovat návrhu autorizační a autentizační infrastruktury pro systémy Adobe Connect a \texttt{meetings.cesnet.cz}. V následující sekci nejprve popisuji stávající autentizační a autorizační infrastrukturu. Dále se v další sekci věnuji analýze požadavků na novou infrastrukturu. V poslední sekci této kapitoly se zabývám samotným návrhem nové autentizační a autorizační infrastruktury, který plyne z provedené analýzy stávajícího systému.    

\section{Popis stávající autentizační a autorizační infrastruktury}
Stávající autentizační a autorizační vrstva webkonferenční infrastruktury sestávající se ze služeb \texttt{meetings.cesnet.cz} a Adobe Connect je založena na systémech Shibboleth \footnote{Shibboleth, \url{www.shibboleth.net}} a Perun \footnote{Perun, \url{www.perun-aai.org}}. Zmíněný Shibboleth je open-source software, který slouží jako autentizační vrstva s jednotným přihlašováním mezi několika systémy. Perun je centralizovaná služba určená pro správu uživatelských údajů. V dosavadní autentizační a autorizační infrastruktuře, byla služba Perun využívána k získání dodatečných informací o autentizovaném uživateli. \par

Představená autentizační a autorizační infrastruktura využívající systémy Shibboleth a Perun je implementována na serveru \texttt{shongo-auth.cesnet.cz}. Je důležité poznamenat, že tento server funguje jako OpenID Connect poskytovatel pro webkonferenční služby \texttt{meetings.cesnet.cz} a Adobe Connect. Zároveň zmíněný server slouží jako klient pro OpenID Connect Proxy IdP v infrastruktuře CESNETu. Uvedený server tedy odbavuje požadavky uživatelů na přihlášení do webkonferenčních systémů \texttt{meetings.cesnet.cz} a Adobe Connect. Po úspěšné autentizaci je uživatel přesměrován na požadované portály. Abychom pochopily změny provedené ve stávající autentizační a autorizační infrastruktuře, je vhodné si detailně představit, jakým způsobem se zpracovávají požadavky na autentizaci uživatelů při přihlašování do zmíněných webkonferenčních systémů.  

\subsection{Autentizační vrstva systému Adobe Connect}
\label{ac-old}
Jak už bylo zmíněno, webkonferenční platforma Adobe Connect využívá k autentizaci a
autorizaci systém Shibboleth. V rámci procesu autentizace si Adobe Connect udržuje interní databázi, ve které si eviduje uživatele. V
uvedené databázi si uchovává informace o uživatelích,
které získá po úspěšné autentizaci uživatele. Pokud záznam o uživateli
v databázi neexistuje, je vytvořen. Jelikož Adobe Connect využívá jen
identifikaci, jméno, příjmení a emailovou adresu autentizovaného uživatele, není v stávající implementaci
potřeba využívat službu Perun. V následujících odstavcích této kapitoly detailně popisuji autentizační a autorizační vrstvu systému Adobe Connect. \par

Zmíněná autentizační a autorizační vrstva webkonferenčního systému Adobe Connect je implementována v adresáři \texttt{/data/app/prod/aclogin-ng/}. Na uvedený adresář se budu dále odkazovat. Konfigurace představované vrstvy se nachází v podaresáři \texttt{config/}. Samotná implementace autentizační a autorizační vrstvy se nachází v podaresáři \texttt{lib/AcLogin/}. \par

Postup při přihlašování do webkonferenčního systému Adobe Connect provozovaného na URL adrese \texttt{www.connect.cesnet.cz} je následující. Pokud se chce uživatel do zmíněného systému přihlásit, klikne na přihlašovací odkaz, který ho přesměruje na URL adresu \texttt{https://shongo-auth.cesnet.cz/aclogin-ng/instance/connect}. Jelikož se jedná o adresu která je chráněna autentizačním systémem Shibboleth, uživatel je před samotným přístupem na uvedenou adresu přesměrován na rozcestník federace identit. Zmíněný rozcestník federace identit se nachází na URL adrese \texttt{https://ds.eduid.cz/wayf.php}. Zde si uživatel vybere poskytovatele identity, pomocí kterého se přihlásí. Po přihlášení u vybraného poskytovatele identity je uživatel  přesměrován zpět na požadovanou URL \texttt{https://shongo-auth.cesnet.cz/aclogin-ng/instance/connect} a začne zpracování dat získaných po autentizaci. Jedná se o uživatelské atributy, které byla nastaveny do proměnných prostředí Apache serveru. Jde o následující atributy dostupné v poli \texttt{\$\_SERVER}: 

\begin{itemize}
    \item \textbf{eppn} představuje unikátní identifikátor uživatele v rámci poskytovatele identit, 
    \item \textbf{mail} je emailová adresa uživatele, 
    \item \textbf{givenName} odkazuje na křestní jméno uživatele,
    \item \textbf{sn} představuje příjmení autentizovaného uživatele.
\end{itemize}
\label{item:adobe-connect}
Po nastavení zmíněných uživatelských atributů do proměnné prostředí se spustí hlavní metoda \texttt{\_run} PHP skriptu \texttt{Application.php} umístěném v adresáři \texttt{lib/AcLogin/}. Prvním krokem v uvedené metodě je inicializace konfigurace. Po té následuje vytvoření objektu představujícího uživatele. Tento objekt je vytvořen pomocí třídy \texttt{AcLogin\_RemoteUser} z popisovaných uživatelských atributů. Dalším krokem je inicializace API objektu, usnadňující manipulaci s interní databází uživatelů systému Adobe Connect. Předtím než začne vytvořený API objekt databázové rozhraní Adobe Connect využívat, provede se pomocí tohoto objektu přihlášení administrátora do databázového systému. \par 

Další fází v představovaném PHP skriptu je využití popisovaného Adobe Connect API objektu ke kontrole, zda uživatel, který se do systému přihlašuje již v interní databázi existuje. Představená kontrola existence uživatele se provede vůči jeho unikátnímu identifikátoru \textbf{eppn}. Pokud uživatel v databázi Adobe Connect neexistuje, je vytvořen a uložen do databáze. Jestliže uživatel v databázi existuje algoritmus pokračuje dále. Pokud je navíc povolena dodatečná aktualizace uživatelských údajů při přihlášení, aktualizují se uživatelovi údaje v databázi Adobe Connect. \par

Po té následuje část zdrojového kódu, v které je provedeno samotné přihlášení uživatele uživatele pomocí unikátního identifikátoru \textbf{eppn} a generovaného hesla. Zmíněné heslo se generuje pomocí hašovací funkce MD5\footnote{Algoritmus hašovací funkce MD5, \url{www.tools.ietf.org/html/rfc1321}}, které se na vstup předá řetězec složený z \textbf{eppn} uživatele a náhodného neměnného řetězce, tzv. soli. Následuje kontrola úspěšnosti přihlášení a vytvoření řetězce sezení, tzv. \textbf{Session String}. Představený \textbf{Session String} jednoznačně identifikuje provedenou autentizaci. Pokud přihlášení proběhlo v pořádku, autentizovaný uživatel je s URL parametrem \textbf{Session String} přesměrován na svoji úvodní stránku webkonferečního systému Adobe Connect.      

\subsection{Autentizační vrstva systému meetings.cesnet.cz}
\label{meetings-old}
Podobně jako autentizační vrstva systému Adobe Connect je i autentizační vrsrva rezervačního systému \texttt{meetings.cesnet.cz} implementována na serveru \texttt{shongo-auth.cesnet.cz}. Systém \texttt{meetings.cesnet.cz} využívá při přihlašování uživatelů jak Shibboleth, tak webovou službu Perun. Je důležité poznamenat, že autentizační vrstva systému \texttt{meetings.cesnet.cz} využívá moduly \textbf{shongo-authn-server} a \textbf{zf2-openid-connect-server-module} k zpracování autentizačních požadavků. Uvedené moduly se nacházejí v adresáři \texttt{/data/app/prod/shongo-authn-server/}. Ačkoliv se jedná o implementaci OpenID Connect na straně poskytovatele služeb, nejsou moduly napojeny na autorizační server OpenID Connect, ale pouze na systém Shibboleth v autentizační infrastruktuře CESNETu. \par

Postup při přihlašování uživatele do systému \texttt{meetings.cesnet.cz} je následující. Pokud se chce uživatel autentizovat, klikne na přihlašovací odkaz, který ho přesměruje na URL adresu \texttt{shongo-auth.cesnet.cz/authn/oic/authorize}. Požadavek na přístup na uvedenou URL adresu obsahuje další parametry nezbytné pro dokončení autentizace:
\label{shongo:state}
\begin{itemize}
    \item \textbf{client\_id} má hodnotu \texttt{meetings.cesnet.cz} a představuje identifikace klienta u autorizačního serveru,
    \item \textbf{redirect\_url} URL adresa na kterou prohlížeč přesměrován po autentizaci, má hodnotu \texttt{https://meetings.cesnet.cz/login}, 
    \item \textbf{state} hodnota sloužící k udržení stavu mezi přihlašovacím požadavkem a odpovědí na autentizaci, 
    \item \textbf{scope} obsahuje hodnotu \textbf{openid}, která značí že se jedná autentizační požadavek respektující pravidla protokolu OpenID Connect,
    \item \textbf{response\_type} definuje způsob průběhu získávání tokenů, v popisovaném požadavku má hodnotu \textbf{code}, což značí že se jedná o tzv. Authorization Code Flow \footnote{OpenID Connect Authorization Code Flow \url{https://openid.net/specs/openid-connect-core-1_0.html\#CodeFlowAuth}},
    \item \textbf{prompt} doplňková informace autorizačnímu serveru, požadující zobrazení autentizačního formuláře koncovému uživateli, i pokud je přihlášen. 
\end{itemize}


Představená URL adresa je namapována na metodu \texttt{authorizeAction} třídy \texttt{AuthorizeController}, která se nachází v modulu \textbf{zf2-openid-connect-server-module}. Ve zmíněném modulu  se nacházejí i následující představované třídy. V uvedené metodě \texttt{authorizeAction} začíná běh autentizačního algoritmu. \par 

Dále je v metodě \texttt{authorizeAction} provedena kontrola, zda je uživatel již přihlášen. Pokud tato skutečnost nastala, předá se řízení algoritmu metodě \texttt{responseAction} nacházející se v aktuální třídě \texttt{AuthorizeController}. Uvedená metoda načte a  aktualizuje data udržované v kontextu o přihlášeném uživateli. Jedná se o objekt \textbf{session} a \textbf{authorize code}. Uvedený objekt \textbf{session} jednoznačně identifikuje vytvořené sezení. \textbf{Authorize code} je textový řetězec, který po autentizaci obdrží klient od autorizačního serveru. Zmíněný \textbf{Authorize code} slouží k jednorázové výměně za tzv. \textbf{ID Token} nebo \textbf{Access Token}. Dalšímu použití uvedených žetonů se detailněji zabývám v rámci popisu technologií \hyperref[sec:oauth]{OAuth} a \hyperref[sec:oidc]{OpenID Connect}. S retězcem \textbf{Authorize Code} je uživatel přesměrován na úvodní stránku přihlášeného uživatele portálu \texttt{meetings.cesnet.cz}. Metodě \texttt{responseAction} se dále věnuji v \hyperref[sec:responseAction]{odstavci níže}. \par 

Jestliže uživatel autentizovaný ještě není, je přesměrován na URL adresu \texttt{shongo-auth.cesnet.cz/authn/oic/authn/shibboleth}. Jelikož je v nastavení Apache uvedená URL adresa vedena jako chráněná systémem Shibboleth, je uživatel přesměrován na již popisovaný rozcestník federace identit. Po přihlášení u vybraného poskytovatele identit se nastaví základní uživatelské atributy do proměnných prostředí Apache serveru. Zmíněné uživatelské informace jsou přístupné v poli \texttt{\$\_SERVER} v atributech \textbf{epp}, \textbf{mail}, \textbf{givenName} a \textbf{sn}. Jde tedy o \hyperref[item:adobe-connect]{stejné uživatelské atributy o jako v případě autentizace do systému Adobe Connect}. Dále je provedeno přesměrování zpět na URL adresu \texttt{shongo-auth.cesnet.cz/authn/oic/authn/shibboleth}, kde se již spustí metoda \texttt{authenticateAction} třídy \texttt{ShibbolethController}. \par

V představené metodě \texttt{authenticateAction} je nastaven objekt \texttt{ShongoAuthn\textbackslash User\textbackslash User} pomocí získaných atributů z proměnných prostředí Apache serveru. Uvedený objekt představuje autentizovaného uživatele a drží si všechny získané informace o daném uživateli. Dále je tento objekt uložen do kontextu proměnné \texttt{\$\_SESSION}. Hodnoty uložené v proměnné \texttt{\$\_SESSION} jsou uchovávané v paměti PHP serveru a jsou sdíleny mezi jednotlivými HTTP požadavky. Pokud všechny předchozí kroky proběhly úspěšně, je předáno řízení algoritmu metodě \texttt{responseAction} nacházející se v třídě \texttt{AuthorizeController}. \par

\label{sec:responseAction}
V metodě \texttt{responseAction} se spouští klíčová metoda \texttt{dispatch} třídy \texttt{Authorize} v které pokračuje algoritmus. Zmíněná metoda \texttt{dispatch} používá metodu \texttt{populateUser} třídy \texttt{PerunWS}, která implementuje rozhraní \texttt{DataConnectorInterface}. \par

Uvedená třída \texttt{PerunWS} se nachází v modulu \textbf{shongo-authn-server} a jde o datový konektor, jenž se připojuje k externí službě. V metodě \texttt{populateUser}, kterou implementuje třída \texttt{PerunWS} se získávají dodatečné informace o autentizovaném uživateli pomocí API webové služby Perun. Nejprve algoritmus z \textbf{eppn} zjistí unikání identifikátor \textbf{perun\_id} v kontextu Perun služby. Potom následuje další dotaz na Perun službu, kterým se algoritmus dotazuje na dodatečné informace o autentizovaném uživateli pomocí \textbf{perun\_id}. Jsou to informace o uživatelově telefonu, příslušnosti k organizaci, jazyku, časové zóně. Dále je získáno pole unikátních identifikátorů osoby v rámci federace dané federace, tzv. \textbf{principal\_names}. Všechny takto získané údaje o uživateli z obdržené odpovědi jsou popsány v příloze \hyperref[table:perunws:user]{A.1}. Uvedené uživatelské informace jsou uloženy do již popisovaného objektu třídy \texttt{ShongoAuthn\textbackslash User\textbackslash User}. \par

Po té se předá řízení algoritmu zpět do metody \texttt{dispatch} v třídě \texttt{Authorize}. V této metodě je dále z získaných uživatelských dat vytvořeno tzv. \textbf{session} a \textbf{Authorize code}. Uvedené \textbf{session} je objekt, reprezentující jedinečné spojení získané autentizací uživatele. Ostatní níže představené pojmy včetně \textbf{Authorize code} jsou definovány v rámci popisu technologií \hyperref[sec:oauth]{OAuth} a \hyperref[sec:oidc]{OpenID Connect}. Uvedené \textbf{session} a \textbf{Authorize code} jsou uloženy do interní databáze serveru \texttt{shongo-authn.cesnet.cz}. Tyto uložené údaje jsou později využity při validaci požadavku na získání \textbf{Access Tokenu} a dalších uživatelských informací. Nakonec je uživatel s retězcem \textbf{Authorize Code} přesměrován na úvodní stránku přihlášeného uživatele portálu \texttt{meetings.cesnet.cz}. \par

Na základě HTTP GET požadavku s přiloženým \textbf{Authorize code} je po té autentizovanému klientu, našem případě službě \texttt{meetings.cesnet.cz}, vydán  \textbf{Access Token}. Zdroj poskytující \textbf{Access Token}, tzv. \textbf{Token Endpoint} je přítomný na URL adrese \texttt{shongo-auth.cesnet.cz/authn/oic/token}. \par 

Obdobně jsou autentizovanému klientu poskytnuty dodatečné uživatelské údaje na základě přiloženého \textbf{Authorize code}. Zdroj který popisované uživatelské údaje drží, tzv. \textbf{UserInfo Endpoint} je dostupný na URL adrese \texttt{shongo-auth.cesnet.cz/authn/oic/userinfo}.  \par 

Následující \hyperref[fig:shongoAuthnProcess]{diagram} znázorňuje přehlednou formou výše popsané procesy v stávající autentizační a autorizační vrstvě systému Shongo nasazeném na portálu \texttt{meetings.cesnet.cz}. 

\begin{figure}[H]
\caption{Schéma komunikace při zpracování autentizačního požadavku do systému Shongo}
\centering
\includegraphics[width=12.8cm]{pics/shongoAuthSystem} 
\label{fig:shongoAuthnProcess}
\end{figure}
\par 


\section{Analýza požadavků}
Z konzultací s vedoucím diplomové práce vyplynuly následující požadavky. Jak už bylo uvedeno v úvodu, hlavní podstatou této diplomové práce je vytvoření nové autentizační a autorizační infrastruktury pro portály \texttt{connect.cesnet.cz} a \texttt{meetings.cesnet.cz} provozované sdružením CESNET. Požadavkem bylo, aby tato nová autentizační a autorizační infrastruktura byla implementována v souladu se specifikací OpenID Connect  \footnote{OpenID Connect specifikace, \url{www.openid.net/specs/openid-connect-core-1_0.html}}. Nezbytnou součástí implemtace bylo, aby autentizační a autorizační infrastruktura byla navázána na Proxy IdP OpenID Connect, které provozuje sdružení CESNET\footnote{Proxy IdP OpenID Connect, \url{www.login.cesnet.cz/oidc/}}. Proxy IdP je komponenta mezi poskytovatelem služeb a identit, která umožňuje systému aby byl dostupný prostřednictvím různých poskytovatelů identit. \par

Dalším požadavkem bylo, aby implementace byla z důvodu centralizace autentizanční a autorizační infrastruktury integrována do stávajícího serveru \texttt{shongo-auth.cesnet.cz}. V neposlední řadě bylo požadováno odstranění nepotřebného kódu po nevyužitých voláních webové služby Perun. Informace o přihlášeném uživateli, které byly dříve zprostředkovány voláním webové služby Perun musely být nově získány z autorizačního serveru poskytovatele OpenID Connect, který se nachází v infrastruktuře CESNETu.     


\section{Návrh nové autentizační infrastruktury}
Zde se budu věnovat otázce, jakým způsobem je koncipován návrh nové autentizační a autorizační infrastruktury pro systémy Adobe Connect a Shongo. Společným prvkem pro oba zmíněné systémy je fakt, že návrh nové autentizační a autorizační architektury navazuje na architekturu stávající. 

\subsection{Autentizační vrstva systému Adobe Connect}
Nová autentizační a autorizační vrstva webkonferenčního systému Adobe Connect bude využívat již existující implementaci a bude s ní úzce svázána. To znamená, že bude dodána implementace do stávajícího autentizačního serveru \texttt{shongo-authn.cesnet.cz}. Jak už bylo uvedeno výše, stávající autentizační a autorizační vrstva využívá systému Shibboleth, který bude v rámci této diplomové práce nahrazen implementací systému OpenID Connect. To znamená, že bude muset být systém OpenID Connect a moduly na straně autentizačního serveru \texttt{shongo-authn.cesnet.cz} do sebe vhodným způsobem integrovány. Tuto skutečnost dosáhneme řádnou instalací autentizačního modulu OpenID Connect webovéo serveru Apache na autentizační server. Dále musíme zajistit, aby došlo k přemapování atributů získaných po autentizaci uživatele. Implementaci zmíněného přemapování popisuji v kapitole \hyperref[ACImpl]{Implementace autentizačního a autorizačního modulu pro systém Adobe Connect}. 

\subsection{Autentizační vrstva systému meetings.cesnet.cz}
Obdobně jako v případě systému Adobu Connect, bude nová implementace autorizace a autentizace uživatelů úzce propojena se stávající implementací. Jak už bylo zmíněno výše, stávající implementace se nachází na serveru \texttt{shongo-auth.cesnet.cz}. Analogicky bude muset být v rámci instalace Apache modulu OpenID Connect provedena konfigurace webového serveru. \par
V další fázi bude na serveru \texttt{shongo-auth.cesnet.cz} následovat implementace procesu, který zajistí propojení s autorizačním serverem OpenID Connect. Uvedená implementace zaručí mapování nově získaných uživatelských dat do stávajícího mechanizmu autentizace a autorizace uživatele. Ta bude realizována pomocí nově vytvořeného datového konektoru OIDC. Je důležité poznamenat, že nová implementace autorizačního algoritmu zachová nezbytnou kontrolu příslušnosti uživatele do skupiny einfra. 
\par

Dalším krokem je adaptace systému \texttt{meetings.cesnet.cz} na provedené změny v rámci implementace na straně serveru \texttt{shongo-auth.cesnet.cz}. Tuto adaptaci představují tři klíčové body. Prvním je nutnost změny databázového schématu týkající se tabulek v kterých si systém \texttt{meetings.cesnet.cz} drží identifikátory svých uživatelů. Jedná se o zvětšení velikosti sloupců držících uživatelské identifikátory, jejichž velikost není v rámci budoucí implementace dostačující. Příčinou této změny je nahrazení identifikátoru uživatele \textbf{perun\_id} za \textbf{einfra id}, jehož délka by překračovala stávající velikost sloupce. Zmíněné identifikátory jednoznačně identifikují uživatele v rámci služby Perun, respektive v OpenID Connect infrastruktuře CESNETu. Tato změna vychází z nutnosti zbavit se závislosti autentizační a autorizační vrstvy na webové službě Perun. \par

Druhým bodem je nutnost uložení dodatečných informací o autentizovaném uživateli do databáze systému \texttt{meetings.cesnet.cz}. Zmíněná data jsou ve stávající implementaci získávána při každém požadavku na zobrazení uživatelských informací pomocí služby Perun. Jelikož nová implementace nebude závislá na webové službě Perun, musí se představené uživatelské informace ukládat do databáze již v rámci procesu autentizace uživatele. Uvedená modifikace implikuje další změnu, a to že bude muset být vytvořen algoritmus, který do databáze systému \texttt{meetings.cesnet.cz} uloží uživatelské informace všech stávajících uživatelů. Obdobným způsobem bude odstraněna vazba na správu skupin pomocí služby Perun. Díky odstranění závislosti autentizační a autorizační vrstvy na webové službě Perun dojde k nemalé snížení složitosti stávajících systémů. Implementaci uvedených změn popisuji v kapitole \hyperref[ShongoImpl]{implementace autentizačního a autorizačního modulu pro systém \texttt{meetings.cesnet.cz}}. \par 

Posledním bodem je modifikace samotného zdrojového kódu systému Shongo. Ta se bude sestávat z adaptace na změnu URL adresy Token a UserInfo Endpointů. Dále bude pozměněna kontrola příslušnosti uživatele do skupin, na základě které je uživateli umožněno využívat zdroje. Uvedená kontrola dříve využívala volání služby Perun, v nové implementaci bude se budou získávat potřebné informace o příslušnosti uživatele do skupin z UserInfo Endpointu. V neposlední řadě bude provedeno refaktorování metod, které využívají API služby Perun. Hlavním smyslem zmíněného refaktorování, bude pozměnění metod, tak aby získavaly požadovaná data z databáze, místo vytváření HTTP požadavků na službu Perun.

\chapter{Použité technologie}
V této kapitole popisuji technologie, které jsem použil při implementaci diplomové práce. Popis doplňuji informací o tom, v které konkrétní části implementace jsem danou technologii použil.
\section{OAuth}
\label{sec:oauth}
OAuth 2.0 \footnote{OAuth 2.0 RFC, \url{www.tools.ietf.org/html/rfc6749}} je autorizační framework, který umožňuje aplikacím třetí strany získat omezený přístup k HTTP službám. Tento framework je tedy určen pro bezpečné delegování přístupu. Zmíněný framework představuje autorizační vrstvu, která odděluje role klienta od držitele zdrojů. Specifikace OAuth 2.0 popisuje následující role: 
\begin{itemize}
    \item \textbf{Resource Owner} je entita, schopná dát souhlas s přístupem k požadovanému zdroji, většinou koncový uživatel,
    \item \textbf{Resource Server} představuje server, uchovávající chráněné zdroje, schopný odpovídat požadavkům přistupujícím k chráněným zdrojům,
    \item \textbf{Client} je aplikace, vytvářející požadavky na získání chráněných zdrojů se svolením role \textbf{Resource Owner},
    \item \textbf{Authorization Server} je server, vydávající \textbf{Access Token} pro roli \textbf{Client} po úspěšném ověření identity \textbf{Resource Owner} a získání oprávnění.
\end{itemize}
Uvedený pojem \textbf{Access Token} je pověření používané k získání přístupu k chráněným zdrojům. \textbf{Access Token} se obvykle vydává ve formátu \textbf{JWT (JSON Web Token)}. Formátu \textbf{JWT} se věnuji v následující sekci v rámci popisu technologie \hyperref[sec:oidc]{OpenID Connect}.  \par

Schéma komunikace v protokolu OAuth 2.0 ilustruje následující obrázek \hyperref[fig:oauth]{3.1}. 

\begin{figure}[H]
\caption{Schéma komunikace v protokolu OAuth 2.0}
\centering
\includegraphics[width=12.8cm]{pics/diplomkaOauth} 
\label{fig:oauth}
\end{figure}
\par 

V prvním kroku (A) \textbf{Client} požaduje autorizaci po \textbf{Resource Owner}. V další fázi (B) \textbf{Client} získá pověření, které představuje oprávnění vlastníka k přístupu ke zdrojům. V následujícím kroku (C) \textbf{Client} požaduje \textbf{Access Token} po \textbf{Authorization Server}, kterému prezentuje udělená oprávnění. Další krok (D) představuje autentizaci a validaci uděleného oprávnění, na základě jehož je vydán \textbf{Access Token}. V předposlední fázi (E) \textbf{Client} přistupuje k \textbf{Resource Server} s \textbf{Access Tokenem} a požadavkem na získání chráněného zdroje. Po validaci \textbf{Access Tokenu} poskytne \textbf{Resource Server} požadované zdroje entitě \textbf{Client} (F). \par 

Autorizační protokol OAuth 2.0 není navržen pro autentizaci uživatelů a neposkytuje možnost získání dalších informací o přihlášeném uživateli. Z těchto důvodů byl protokol OAuth 2.0 v implementaci použit spolu s technologií OpenID Connect, která zmíněnou funkcionalitu definuje a umožňuje ji vhodně implementovat. 

\section{OpenID Connect}
\label{sec:oidc}
OpenID Connect 1.0 \footnote{Dokumentace OpenID Connect 1.0, \url{www.openid.net/specs/openid-connect-core-1_0.html}} je protokol, který umožňuje aplikaci verifikovat identitu uživatele na základě autentizace provedené autorizačním serverem. Jedná se o autentizační vrstvu nad frameworkem OAuth 2.0. Důležitou vlastností OpenID Connect je definice způsobu získaní základních informací o přihlášeném uživateli. \par

Hlavním rozšířením protokolu OpenID Connect oproti OAuth 2.0 umožňující autentizaci je definice nové datové struktury \textbf{ID Token}. Uvedený \textbf{ID Token} je bezpečnostní žeton, který obsahuje informace o autentizaci a případná další doplňková data koncového uživatele. Zmíněné informace nacházející se ve struktuře \textbf{ID Token} nazýváme v tomto kontextu \textbf{Claims}. Popisované \textbf{Claims} vždy nesou informace ve formátu dvojice název \textbf{Claim} a hodnota \textbf{Claim}. Představované \textbf{Claims}, které uchovávají informace o autentizované entitě, jsou ve struktuře \textbf{ID Token} zakódovány ve formátu \textbf{JWT} (JSON Web Token)\footnote{JSON Web Token RFC, \url{www.tools.ietf.org/html/rfc7519}}.    \par

\textbf{JWT} je textový řetězec reprezentující \textbf{Claims} jako JSON objekt\footnote{JavaScript Object Notation RFC, \url{https://tools.ietf.org/html/rfc7159}} zakódovaný pomocí Base64\footnote{Base64 kódování, \url{www.tools.ietf.org/html/rfc4648}} kódování a HMAC s SHA-256\footnote{HMAC s SHA-256, \url{ www.tools.ietf.org/html/rfc4868}}.  Popisovaný \textbf{JWT} je standardně rozdělen na části hlavičku a tělo. Po aplikaci algoritmů Base64 a HMAC s SHA-256 na tyto dvě části se vytvoří výsledný řetězec. Konkrétní mechanizmus způsobu vytváření \textbf{JWT} je detailně popsán v \href{http://www.tools.ietf.org/html/rfc7519}{odkazovaném RFC 7519} . \par

Protokol OpenID Connect dále definuje jaké \textbf{Claims} musí struktura \textbf{ID Token} obsahovat:

\begin{itemize}
    \item \textbf{iss} představuje identifikátor entity, která vytvořila \textbf{ID Token}, 
    \item \textbf{sub} označuje identifikátor autentizovaného uživatele, 
    \item \textbf{aud} je pole identifikátorů entit, pro které je vydaný \textbf{ID Token} určen,
    \item \textbf{exp} reprezentuje časový údaj, do kdy je vydaný \textbf{ID Token} platný
    \item \textbf{iat} označuje informaci, kdy byl \textbf{ID Token} vydán.
\end{itemize}
\par

Před samotným rozborem schématu komunikace v protokolu OpenID Connect je vhodné definovat následující pojmy. \textbf{OpenID Provider (OP)} představuje autorizační server schopný autentizovat koncového uživatele. Jednou z klíčových funkcí \textbf{OP} je možnost poskytnout \textbf{Relying Party} \textbf{Claims} o autentizační události a autentizovaném uživateli. Zmíněný pojem \textbf{Relying Party (RP)} označuje klientskou aplikaci, požadující autentizaci uživatele a \textbf{Claims} od \textbf{OP}. \par
Následující obrázek \hyperref[fig:oidc]{3.2} popisuje s použitím dříve definovaných pojmů schéma komunikace v protokolu OpenID Connect.

\begin{figure}[H]
\caption{Schéma komunikace v protokolu OpenID Connect 1.0}
\centering
\includegraphics[width=12.8cm]{pics/diplomkaOIDC} 
\label{fig:oidc}
\end{figure}
\par 

V první fázi (A) pošle \textbf{Relying Party} autentizační požadavek autorizačnímu serveru \textbf{OpenId Provider}. V dalším kroku (B) \textbf{OP} autentizuje koncového uživatele a provede autorizaci. V třetí fázi (C) odpoví autorizační server \textbf{OP} na požadavek informací o výsledku autentizace spolu s žetony \textbf{ID Token} a \textbf{Access Token}. V dalším kroku (D) \textbf{RP} může pomocí \textbf{Access Tokenu} poslat \textbf{OP} požadavek na získání uživatelských dat. Nakonec poslední fází (E) komunikačního schématu je odpověď \textbf{OP} s \textbf{Claims} obsahující uživatelská data, které si v předchozím kroku \textbf{RP} vyžádal. \par

V předchozím odstavci je zmíněna předposlední fáze \textbf{D} ve které \textbf{RP} po autentizaci posílá požadavek \textbf{OP} na získání dat o uživateli. Zdroj, který tyto informace uchovává se nazývá \textbf{UserInfo Endpoint}. Představovaný \textbf{UserInfo Endpoint} je chráněná URL adresa, která akceptuje HTTP požadavky pouze s přiloženým validním \textbf{Access Tokenem} získaným při autentizaci. Formát odpovědi na tyto dotazy se nastavuje při registraci \textbf{OpenID Providera}.  \par

Jak už bylo zmíněno v sekci která se věnuje popisu stávající implementace systému \texttt{meetings.cesnet.cz}, proces autentizace a autorizace je řízen pomocí tzv. Authorization Code flow. Tento způsob získání \textbf{ID Tokenu} a \textbf{Access Tokenu} vkládá do popisovaného komunikačního schématu v protokolu OpenID Connect další fázi. Místo zmíněných žetonů je v kroku (C) autorizačním serverem \textbf{OP} vrácen vygenerovaný textový řetězec tzv. \textbf{Authorize code}. Tento \textbf{Authorize code} je v následujícím kroku vyměněn za požadované žetony \textbf{ID Token} a \textbf{Access Token}. Podobně jako v případě \textbf{UserInfo Endpoint}, chráněný zdroj na straně autorizačního serveru \textbf{OP}, který poskytuje žetony na základě validace \textbf{Authorize code} se nazývá \textbf{Token Endpoint}.

\par
OpenID Connect je klíčová technologie, který byla využita k vytvoření autentizační a autorizační nové infrastruktury při vypracování této diplomové práce. Existuje velké množství certifikovaných implementací technologie OpenID Connect v mnoha různých programovacích jazycích\footnote{Certified OpenID Connect implementations, \url{www.openid.net/developers/certified/}}. V této diplomové práci jsem použil MITREid Connect\footnote{MITREid Connect, \url{www.mitreid-connect.github.io/}}, jelikož je již integrována v infrastruktuře CESNETu jako implementace OpenID Connect. Jedná se o webovou aplikaci napsanou v programovacím jazyce Java, která je založená na frameworku Spring Security\footnote{Spring Security framework documentation, \url{www.docs.spring.io/spring-security/site/docs/current/reference/htmlsingle/}}. 

\section{Ostatní technologie}
V této kapitole popisuji další technologie, které jsem využil v rámci implementace autentizační a autorizační infrastruktury. 
\par
Jednou z hlavních technologií, kterou jsem použil při vypracování této diplomové práce je webový server Apache. Ústředním modulem, který slouží pro integraci s poskytovatelem OpenID Connect je \textbf{mod\_auth\_openidc}\footnote{Apache modul mod\_auth\_openidc, \url{https://github.com/zmartzone/mod\_auth\_openidc}}. Zmíněný modul umožňuje autentizovat a autorizovat klientskou aplikaci vůči autorizačnímu serveru OpenID Connect. Modul dále může sloužit i pro nastavení serveru, který funguje jako tzv. \textbf{Resource Server} v rámci protokolu OAuth 2.0. Konkrétní způsob konfigurace Apache spolu s OpenID Connect modulem popisuji v sekci \hyperref[apacheConfig]{Konfigurace Apache}.

\par
Další klíčovou technologií, kterou jsem při práci využil je programovací jazyk PHP. V uvedeném programovacím jazyce jsem na straně serveru \texttt{shongo-auth.cesnet.cz} implementoval autentizační a autorizační logiku. Architektura serveru \texttt{shongo-auth.cesnet.cz} starající se o autentizaci a autorizaci uživatelů je vystavěna na aplikačním frameworku Zend. \footnote{Zend Framework, www.framework.zend.com}. Zmíněná implementace je detailněji popsána v kapitolách Implementace autentizační a autorizační vrstvy systémů \hyperref[ACImpl]{Adobe Connect} a \hyperref[ShongoImpl]{meetings.cesnet.cz}. 

\par

Další technologií, kterou jsem při při vývoji nové autentizační a autorizační infrastruktury použil je databázový server PostgreSQL \cite{postgresql}. S PostgreSQL jsem pracoval při adaptaci databáze systému \texttt{meetings.cesnet.cz} na provedené v změny autentizační a autorizační vrstvě. Zmíněná adaptace sestávala z následujících dílčích kroků. Prvním bylo nahrazení primárního klíče uživatele z \textbf{perun\_id} na unikátní identifikátor uživatele  \textbf{einfra id} v rámci infrastruktury OpenID Connect CESNETu. Další krok tvořil vkládání dodatečných informací o stávajících uživatelích rezervačního systému \texttt{meetings.cesnet.cz}. V neposlední řadě jsem představený databázový server využil při modifikaci stávajícího schématu databáze systému \texttt{meetings.cesnet.cz}. Tyto operace detailněji popisuji v sekci \hyperref[ShongoImpl-web]{4.3.2}. 
\par 
Posledními významnými technologiemi, které jsem využil při vypracování této diplomové práce jsou programovací jazyk Java a objektově relační framework Hibernate \footnote{Hibernate ORM framework , \url{www.hibernate.org/orm/}}. Jedná se o technologie, které jsou použity ve stávající implementaci rezervačního systému Shongo a využil jsem je při integraci s novou autentizační a autorizační vrstvou.  


\chapter{Popis implementace}
V této kapitole seznamuji s detailním postupem při implementaci a instalaci nové autentizační a autorizační infrastruktury využívající protokol OpenID Connect. V kapitole \hyperref[apacheConfig]{4.1} popisuji jakým způsobem probíhala instalace a konfigurace modulu OpenID Connect pro webový server Apache. V dalších kapitolách \hyperref[ACImpl]{4.2} a \hyperref[ShongoImpl]{4.3} se věnuji implementaci autentizační a autorizační vrstvy systému Adobe Connect, respektive systému Shongo.   


\section{Instalace a konfigurace Apache modulu pro OpenID Connect}
\label{apacheInstall}
V této kapitole popisuji instalaci a konfiguraci Apache modulu pro OpenID Connect. Dále se věnuji registraci klienta \texttt{shongo-auth.cesnet.cz} do ProxyIdP infrastruktury CESNETu. Zmíněné operace byly provedeny podle pokynů dostupných v prezentaci prezentované na dnech s Perunem \cite{oidcConfigPresentation}.

\label{apacheConfig}
\subsection{Instalace knihoven modulu mod\_auth\_openidc}
Instalace knihoven nutných pro zprovoznění Apache modulu OpenID Connect byly v rámci diplomové práce prováděny na serveru \texttt{shongo-auth.cesnet.cz} s operačním systémem Debian GNU/Linux 7 (wheezy). 
\par 
Autentizační a autorizační modul, který slouží pro zprovoznění serveru jako tzv. \textbf{OpenID Connect Relying Party} se nazývá \textbf{mod\_auth\_openidc\footnote{OpenID Connect Relying Party pro Apache server \url{https://github.com/zmartzone/mod\_auth\_openidc}}}. Důležitým předpokladem pro instalaci představeného modulu je již dříve nainstalovaný server Apache. Apache se na serveru \texttt{shongo-auth.cesnet.cz} již nachází a tudíž jej nebylo nutné instalovat. 
\par 
Prvním krokem, který jsem provedl byla instalace nezbytných knihoven \textbf{libjansson4} a \textbf{libhiredis0.13}, \textbf{libcjose0\_0.5} a \textbf{libapache2-mod-auth-openidc\_2.3.3-1}. Po instalaci uvedených knihoven následovala registrace klienta do Proxy IdP infrastruktury. 

\subsection{Registrace klienta do Proxy IdP infrastruktury}

Po úspěšně dokončené instalaci uvedených knihoven následovala registrace klienta \texttt{shongo-auth.cesnet.cz} do Proxy IdP infrastruktury sdružení CESNET. Zmíněná registrace probíhala ve webové aplikaci MITREid Connect na adrese \url{https://login.cesnet.cz/oidc/}. Po přihlášení do webové aplikace bylo prostřednictvím uživatelského rozhraní v levém menu v kolonce \textbf{Developer} vybrán odkaz \textbf{Self-service client registration}. Uvedená akce přesměrovala prohlížeč na stránku s tlačítkem \textbf{New Client} sloužící pro registraci nového klienta. Po kliknutí na zmíněné tlačítko aplikace přesměrovala prohlížeč na registrační formulář s konfigurací nového klienta. V záložce \textbf{Main} byla vyplněna kolonka \textbf{Redirect URI (s)} adresou \texttt{https://shongo-auth.cesnet.cz/oauth2callback}. Uvedená kolonka představuje URL adresu pro přesměrování po autentizaci koncového uživatele. Dále byly vyplněny údaje v záložce \textbf{Access} v kolonce \textbf{Scope}. Zmíněné \textbf{Scopes} indikují údaje o uživateli, ke kterým bude chtít registrovaný klient po autentizaci přistupovat. V rámci registrace klienta byly vybrány následující \textbf{Scopes}.
\begin{itemize}
    \item \textbf{openid} představuje unikátní identifikátor uživatele v rámci Proxy IdP ifrastruktury, tzv. einfra id,
    \item \textbf{profile} jde o osobní profil uživatele požadující \textbf{Claimy} \textbf{name}, \textbf{given\_name}, \textbf{middle\_name}, \textbf{family\_name}, \textbf{preferred\_username}, \textbf{zoneinfo} a \textbf{locale},
    \item \textbf{email} označuje emailovou adresu uživatele,
    \item \textbf{address} je poštovní adresa uživatele,
    \item \textbf{phone} odkazuje na telefonní číslo uživatele,
    
    \item \textbf{organization} představuje domovskou organizaci ke které uživatel patří,
    \item \textbf{eppns} značí seznam všech \textbf{eduPersonPrincipalName} uživatele, jde o jednoznačné identifikátory v rámci federace, mající formát uživatelské\_jméno@doména,  
    \item \textbf{eduPersonEntitlement} obsahuje seznam skupin, kterých je uživatel členem a jsou na službu \texttt{meetings.cesnet.cz} přiřazeny.
    
\end{itemize}
Dále je důležité zmínit, že v kolonce \textbf{Grant Types} byl zvolen \textbf{authorization code}, který respektuje nastavení předávaní tokenů v protokolu OpenID Connect v autentizační a autorizační architektuře registrovaného serveru \texttt{shongo-auth.cesnet.cz}.  \par

Po uložení zmíněné konfigurace byly aplikací náhodně vygenerovány tyto následující údaje. Jde o \textbf{Client ID}, které jednoznačně identifikuje registrovaného klienta \texttt{shongo-auth.cesnet.cz}. Dále byl vygenerován registrační token, tzv. \textbf{Registration Access Token}, který slouží pro pozdější zpřístupnění editace konfigurace vytvořeného klienta. Posledním údajem, který byl vygenerován je sdílené tajemství mezi klientskou aplikací a autorizačním serverem, tzv. \textbf{Client Secret}. Zmíněný \textbf{Client Secret} autentizuje požadavky klientské aplikace vůči autorizačnímu serveru. Vygenerované údaje jsem dále využil v rámci konfigurace modulu mod\_auth\_openidc, kterou popisuji v následující sekci \hyperref[sec:mod-conf]{4.1.3}.

\subsection{Konfigurace Apache modulu mod\_auth\_openidc} 
\label{sec:mod-conf}
Předpokladem pro kompletní zprovoznění OpenID Connect modulu je spuštění příkazu \texttt{a2enmod auth\_openidc} který zapnul tento dříve nainstalovaný modul. Potom následovala nastavení nainstalovaného modulu v konfiguračním souboru Apache serveru \texttt{/etc/apache2/sites-available/default-ssl}. Do zmíněného souboru jsem přidal následující konfigurace.

\begin{itemize}
    \item \textbf{LoadModule auth\_openidc\_module /usr/lib/apache2\\ /modules/mod\_auth\_openidc.so} značí načtení požadovaného OpenID Connect modulu, 
    \item \textbf{OIDCProviderMetadataURL https://login.cesnet.cz/oidc/.well-known/openid-configuration} představuje adresu odkazu na konfiguraci poskytovatele OpenID Connect, 
    \item \textbf{OIDCProviderMetadataRefreshInterval 3600} definuje nastavení intervalu pro obnovení metadat poskytovatele OpenID Connect, 
    \item \textbf{OIDCClientID client\_id} nastavuje identifikaci klienta, tato hodnota byla vygenerovaná v rámci registrace,
    \item \textbf{OIDCClientSecret client\_secret} definuje sdílené tajemství mezi klientem a poskytovatelem OpenID Connect, také se použije hodnota, která byla dříve vygenerovaná při registraci,
    \item \textbf{OIDCScope "openid email profile address phone organization eppns eduPersonEntitlement"} slouží pro nastavení \textbf{Scopes}, které jsou požadované od poskytovatele OpenID Connect, vyčet \textbf{Scopes} odpovídá zvoleným \textbf{Scopes} při registraci klienta do Proxy IdP infrastruktury,
    \item \textbf{OIDCRedirectURI /oauth2callback} značí adresu na kterou bude prohlížeč přesměrován po autentizaci, musí mít stejnou hodnotu jako zvolená hodnota při registraci klienta, 
    \item \textbf{OIDCCryptoPassphrase randompassword} představuje heslo pro šifrování cookie a cache dat. 
\end{itemize}

Po nastavení uvedené konfigurace, mohlo dojít ke stanovení URL adres, které při přístupu požadují OpenID Connect autentizaci. Zmíněnou ochranu adres pro autentizační a autorizační vrstvu Adobe Connect popisuje pomocí Apache direktiv \textbf{Location}, \textbf{AuthType} a \textbf{Require} následující výňatek z konfiguračního souboru Apache.
TODO vypsat co direktivy znamenají
\begin{lstlisting}
          <Location /oidc-ac>
                  AuthType openid-connect
                  Require valid-user
          </Location>
\end{lstlisting}
\label{ac-location}

Obdobně ochranu URL adres mapovaných na moduly, které se starají o autentizaci a autorizaci uživatelů přistupujících na server \texttt{meetings.cesnet.cz} zajišťuje následující úryvek. 

\begin{lstlisting}
          <Location /oidc-authn/oic/authn/oidc>
                  AuthType openid-connect
                  Require valid-user
          </Location>
          
          <Location /oidc-authn/oic/authorize-response>
                  AuthType openid-connect
                  Require valid-user
          </Location>
\end{lstlisting}
První představená URL adresa \texttt{/oidc-authn/oic/authn/oidc} je mapována na nově vytvořenou třídu \texttt{OIDCController}, která se stará o zprostředkování autentizace. Druhá URL adresa \texttt{/oidc-authn/oic/authorize-response} slouží po autentizaci k zpřístupnění získaných \textbf{Claims} o uživateli. Dalšímu použití  chráněných URL adres se věnuji v rámci popisu implementace autentizační a autorizační vrstvy systému \texttt{meetings.cesnet.cz} v kapitole \hyperref[ShongoImpl-authn]{4.3.1}.

\label{shongo-location}

\section{Implementace autentizační a autorizační vrstvy systému Adobe Connect}
\label{ACImpl}
V této sekci se zabývám popisem implementace, kterou bylo nutné provézt při integraci stávající autentizační a autorizační vrstvy systému Adobe Connect s OpenID Connect Proxy IdP provozované sdružením CESNET. 
\par 
Pro pochopení provedených změn je důležité připomenou ústřední prvky stávající implementace. V stávající implementaci jsou uživatelské atributy po autentizaci prostřednictvím systému Shibboleth mapovány na objekt představujícího uživatele. Dále je důležité zmínit, že jednou z hlavní částí této diplomové práce je nahrazení autentizačního systému Shibboleth za implementaci protokolu OpenID Connect. Jelikož autentizační modul Apache \textbf{mod\_auth\_openidc} zprostředkuje atributy autentizovaného uživatele odlišným způsobem než aktuální konfigurace sytému Shibboleth, bylo klíčové zajistit přemapování zmíněných uživatelských atributů. Popis konfigurace a instalace zmíněného modulu \textbf{mod\_auth\_openidc} je představen v kapitole \hyperref[apacheInstall]{4.1}.  
\par

Popsané přemapování uživatelských atributů bylo provedeno v konfiguračním souboru \texttt{config/connect.ini} autentizační a autorizační vrstvy. Následující výňatek konfiguračního souboru popisuje nové mapování uživatelských atributů do proměnných autentizačního algoritmu. 

\begin{lstlisting}[
    basicstyle=\footnotesize\ttfamily,
    columns=fullflexible,
    commentstyle=\color{gray}\ttfamily,
    morekeywords={},
    otherkeywords={=,:},
    keywordstyle={\color{red}\bfseries},
    backgroundcolor=\color{background},
    frame=lines
]
uid_field = OIDC_CLAIM_sub
eppns_field = OIDC_CLAIM_eppns
mail_field = OIDC_CLAIM_email
givenName_field = OIDC_CLAIM_given_name
sn_field = OIDC_CLAIM_family_name
\end{lstlisting}
První řádek výňatku mapuje unikátní identifikátor einfra id, druhý řádek seznam obsahující \textbf{eduPersonPrincipalName}, další řádek emailovou adresu, předposlední řádek jméno a poslední řádek příjmení uživatele. Atribut \textbf{eppns\_field} značí uživatelské identifikátory v jednotlivých organizacích, kterých je uživatel členem. Uvedené atributy jsou v autentizačním algoritmu nastaveny objektu představujícího uživatele. 
\par 

Se zmíněným objektem představujícího přihlašovaného uživatele se dále pracuje, tak jako ve stávající implementaci. To znamená že je provedena kontrola, zda uživatel již existuje v interní databázi webkonferenčního systému Adobe Connect. Protože ve stávající implementaci atribut \textbf{eppns\_field} obsahoval pouze jedno \textbf{eduPersonPrincipalName}, kontrolovala se existence pouze jednoho přítomného identifikátoru v databázi. Detailní popis představeného procesu kontroly existence uživatele v databázi ve stávající implementaci autentizační a autorizační vrstvy systému Adobe Connect je uveden v kapitole \hyperref[ac-old]{2.1.1}. V nové implementaci musela být provedena kontrola existence uživatelského identifikátoru \textbf{eduPersonPrincipalName} odlišným způsobem. Příčinou této změny je fakt, že do atributu \textbf{eppns\_field} je nově mapován seznam několika \textbf{eduPersonPrincipalName} místo jediného uživatelského identifikátoru.
\par
Z popisovaného důvodu z předchozího odstavce byla do autentizačního algoritmu přidána metoda \texttt{\_getExistingUserIdByEppns}. Tato metoda najde ze seznamu všech \textbf{eppns\_field} takový identifikátor uživatele, který již v interní databázi Adobe Connect existuje. Pokud metoda \texttt{\_getExistingUserIdByEppns} nenajde žádného uživatele v interní databázi Adobe Connect, vytvoří algoritmus v databázi záznam představujícího uživatele. Zmíněný záznam, je primárně vytvořen z takového \textbf{eduPersonPrincipalName}, které představuje uživatelský identifikátor v rámci skupiny einfra. Popisované \textbf{eduPersonPrincipalName} identifikující uživatele v rámci einfra skupiny má formát \textbf{{uživatelský\_identifikátor}@einfra.cesnet.cz}. Důsledkem příslušnosti uživatele do skupiny einfra je tzv. \textbf{LoA} úrovně 2. \textbf{LoA\footnote{Level of Assurance RFC, \url{https://tools.ietf.org/html/rfc6711}} (Level of Assurance)} je úroveň záruky vyjadřující míru spolehlivosti prostředků pro
elektronickou identifikaci při určování totožnosti dané osoby.   Pokud v krajním případě uživatel nepatří do skupiny einfra je v rámci nové implementace použit první uživatelský identifikátor ze seznamu \textbf{eduPersonPrincipalName}. Dále již běh algoritmu pokračuje obdobně jako ve stávající implementaci autentizační a autorizační vrstvy systému Adobe Conect, kterou popisuji v kapitole \hyperref[ac-old]{2.1.1}. 


\par 
V neposlední řadě, je důležité poznamenat, že nová implementace autorizační a autentizační vrstvy systému Adobe Connect byla přesunuta do adresáře \texttt{/app/oidc/}. Dále je důležité zmínit, že chráněná URL adresa definovaná v \hyperref[ac-location]{konfiguračním souboru} Apache je mapována právě na uvedený adresář \texttt{/app/oidc/}. Z toho plyne, že nová URL adresa určená pro zpracování autentizačních požadavků do webkonferenčního systému Adobe Connect má formát \texttt{https://shongo-auth.cesnet.cz/oidc-ac/instance/connect}.


\section{Implementace autentizační a autorizační vrstvy systému meetings.cesnet.cz}
\label{ShongoImpl}
V této kapitole se zabývám popisem postupu implementace při integraci stávající autentizační a autorizační vrstvy služby \texttt{meetings.cesnet.cz} do Proxy IdP provozované sdružením CESNET. V podkapitole \hyperref[ShongoImpl-authn]{4.3.1} se představuji implementaci na straně serveru \texttt{shongo-auth.cesnet.cz}. Kapitola \hyperref[ShongoImpl-authn]{4.3.2} pojednává o implementačních změnách na straně rezervačního systému Shongo, provozovaného na portálu \texttt{meetings.cesnet.cz}. 

\subsection{Implementace autentizační a autorizační logiky na serveru shongo-auth.cesnet.cz}
\label{ShongoImpl-authn}
Zde se věnuji popisu implementace algoritmu, který zajišťuje autentizaci a autorizaci klienta \texttt{shongo-auth.cesnet.cz} vůči autorizačnímu serveru OpenID Connect v infrastruktuře CESNETu.  \par

Jak vyplynulo ze zadání diplomové práce, klíčovou součástí implementace je vytvoření nové autentizační a autorizační vrstvy systému \texttt{meetings.cesnet.cz}. Jednou z hlavních částí zmíněné implementace je nahrazení stávajících autentizačních systémů Shibboleth a Perun za OpenID Connect poskytovatele.

\par

Je důležité zmínit, že o samotnou autentizaci uživatelů se stará konkrétní poskytovatel identit ve spolupráci s autorizačním server OpenID Connect. Delegování autentizace je jednou ze základních vlastností protokolu OpenID Connect a zprovoznění této funkcionality bylo dosaženo řádnou instalací Apache modulu \textbf{mod\_auth\_openidc} a registrací klienta \texttt{shongo-auth.cesnet.cz} do Proxy IdP infrastruktury CESNETu. Proto je autentizace přenechána těmto entitám a není již řešena na straně serveru \texttt{shongo-auth.cesnet.cz}. Uvedené řešení respektuje stávající architekturu která dříve obdobným způsobem využívala k autentizaci uživatelů systém Shibboleth. 

\par 
Dalším krokem vykonaným v rámci implementace nové autentizační a autorizační vrstvy bylo nahrazení funkcionality webové služby Perun.  
Jelikož autentizační algoritmus na straně serveru \texttt{shongo-auth.cesnet.cz} počítá s uživatelskými atributy získanými ze systémů Shibboleth a Perun, muselo dojít k vhodnému přemapování těchto atributů. Systém Shibboleth poskytoval pouze podmnožinu uživatelských atributů, něž jakou poskytuje služba Perun, a tudíž stačilo v tomto kontextu nahradit pouze funkcionalitu zmiňované služby Perun. \par

Jak již bylo uvedeno v popisu stávající autentizační a autorizační logiky systému \texttt{shongo-auth.cesnet.cz}, volání služby Perun se realizují v datovém konektoru PerunWS. Hlavní úkolem tohoto datového konektoru je získání a zpracování dat z webové služby Perun do objektu představujícího uživatele. Zmíněný datový konektor byl nahrazen novým datovým konektorem, který zpracovává uživatelské údaje po autentizaci prostřednictvím poskytovatele OpenID Connect. 
\par

Jak již bylo uvedeno výše, nezbytnou fází implementace autentizace bylo vytvoření nového datového konektoru. Úkolem uvedeného datového konektoru je zpracování uživatelských atributů dostupných v globální proměnné prostředí \texttt{\$\_SERVER}. Je důležité zmínit, že uživatelské atributy jsou po autentizaci uživatele do proměnné \texttt{\$\_SERVER} zprostředkovány díky dříve nainstalovanému Apache modulu \textbf{mod\_auth\_openidc}. Abychom mohli představené uživatelské atributy zpracovat, byl vytvořen soubor \texttt{OIDC.php}, který reprezentuje zmíněný datový konektor. Vytvořený konektor \texttt{OIDC.php} plně nahrazuje konektor \texttt{PerunWS.php}, který v předcházející implementaci získával informace o přihlášeném uživateli pomocí webové služby Perun. Nově vytvořený konektor mapuje následující atributy v poli \texttt{\$\_SERVER} do objektu představujícího autentizovaného uživatele \texttt{ShongoAuthn\textbackslash User\textbackslash User}. 

\begin{itemize}
    \item \textbf{OIDC\_CLAIM\_sub} je unikátní identifikátor uživatele v infrastruktuře CESNETu, tzv. einfra id,
    \item \textbf{OIDC\_CLAIM\_given\_name} představuje křestní jméno uživatele,
    \item \textbf{OIDC\_CLAIM\_family\_name} obsahuje příjmení uživatele,
    \item \textbf{OIDC\_CLAIM\_zoneinfo} značí časovou zónu preferovanou uživatelem, 
    \item \textbf{OIDC\_CLAIM\_locale} identifikuje vybraný jazyk uživatele,
    \item \textbf{OIDC\_CLAIM\_email} představuje emailovou adresu uživatele,
    \item \textbf{OIDC\_CLAIM\_organization} definuje hlavní organizaci ke které uživatel patří
    \item \textbf{OIDC\_CLAIM\_address} značí poštovní adresu uživatele,
    \item \textbf{OIDC\_CLAIM\_eppns} jsou unikátní identifikátory v rámci organizací, tzv. eduPersonPrincipalName, 
    \item \textbf{OIDC\_CLAIM\_eduPersonEntitlement} obsahuje seznam skupin, kterých je uživatel členem a jsou na službu přiřazeny. 
\end{itemize}

Po nastavení objektu představujícího uživatele vypsanými atributy pokračuje autentizační algoritmus stejným způsobem jako v případě stávající implementace v kapitole \hyperref[meetings-old]{2.1.2}.

\par

Dalším fází byla implementace autorizace, v rámci které jsem vytvořil validátor \texttt{OIDCUser.php}. Úkolem vytvořeného validátoru \texttt{OIDCUser.php} je kontrola, zda autentizovaný uživatel patří do skupiny einfra. Logika zmíněného validátoru spočívá v tom, že ověřuje zda je v seznamu skupin \textbf{OIDC\_CLAIM\_eduPersonEntitlement} do kterých uživatel patří přítomen řetězec \texttt{urn:geant:cesnet.cz:group:einfra:einfra\#perun.cesnet.cz}. Pokud uživatel do požadované skupiny patří, běh algoritmu pokračuje stejným způsobem jako v stávající implementaci. Pokud uživatel do skupiny nepatří, je přesměrován na stránku ProxyIdP infrastruktury CESNET s registrací.   

\par

V rámci implementace nové autentizační a autorizační vrstvy na straně serveru \texttt{shongo-auth.cesnet.cz} byly odstraněny třídy zajišťující autentizaci prostřednictvím systému Shibboleth. Dále byla odstraněna logika, která zajišťuje získání dodatečných informací o uživateli pomocí webové služby Perun. Díky provedené implementaci tedy došlo k odstranění závislosti na externím systému a značnému zjednodušení autentizační a autorizační vrstvy serveru \texttt{shongo-auth.cesnet.cz}.

V neposlední řadě je důležité poznamenat, že vývoj nové autentizační a autorizační vrstvy systému Adobe Connect zahrnoval přejmenování stávajících tříd. V rámci těchto operací, jsem vytvořil třídu \texttt{OIDCController}. Zmíněná třída kopíruje chování třídy \texttt{ShibbolethController}. Jde tedy o logiku, která uživatele přesměruje na rozcestník federace identit v infrastruktuře OpenID Connect CESNETu kde proběhne autentizace. K této změně vedla nutnost zachování jmenných konvencí, tak aby název odpovídal chování dotyčné třídy. 

\par 

Na závěr této kapitoly je důležité zmínit, že v rámci implementace byla zkopírována celé autentizační a autorizační vrstva do adresáře \texttt{/app/shongo-authn/}. Přístup k uvedenému adresáři přes webový server Apache je nastaven v konfiguračním souboru \texttt{/etc/apache2/sites-available/default-ssl} pomocí URL adresy \texttt{shongo-auth.cesnet.cz/oidc-authn}. Uvedená URL adresa je chráněna OpenID Connect modulem Apache, který byl nakonfigurován v kapitole \hyperref[shongo-location]{4.1.3}.

\subsection{Zpracování a uložení uživatelských dat získaných z webové služby Perun}
V následujících odstavcích se zabývám popisem modifikací databáze, kterou rezervační systém \texttt{meetings.cesnet.cz} využívá pro svůj provoz.
\par

Je důležité poznamenat, že autorizační server OpenID Connect neumožňuje získat uživatelský identifikátor \textbf{perun\_id} prostřednictvím \textbf{Claims}. Bohužel systém \texttt{meetings.cesnet.cz} s tímto identifikátorem \textbf{perun\_id} pracuje, a proto byla nutná náhrada zmíněného identifikátoru za \textbf{einfra id}. Zmíněné náhradě identifikátorů předcházela modifikace databázového schématu, protože nový identifikátor \textbf{einfra id} by překračoval maximální povolenou délku sloupce vyhrazenou pro \textbf{perun\_id}. Uvedený sloupec s uživatelským identifikátorem se používal v tabulkách \textbf{person}, \textbf{user\_settings}, \textbf{resource}, \textbf{acl\_identity} a \textbf{abstract\_reservation\_request}. Jelikož je databázové schéma definované pomocí technologie Hibernate, došlo ke změně délky sloupce jak na straně databáze, tak na straně třídy definující vlastnosti používaných tabulek. 
\par  
Na straně zdrojového kódu jsem provedl změnu délky sloupce v souboru \texttt{cz/cesnet/shongo/controller/api/Controller.java}. Uvedená změna se týkala proměnné \texttt{USER\_ID\_COLUMN\_LENGTH}, která definuje délku sloupce. Sloupec je typu VARCHAR a byl změněn z délky 32 na 255. Dále byla provedena změna délky sloupce v databázových tabulkách pomocí následujícího SQL příkazu.

\begin{lstlisting}[
           language=SQL,
           showspaces=false,
           basicstyle=\footnotesize\ttfamily,
           commentstyle=\color{gray},
           backgroundcolor=\color{background},
           frame=lines
        ]
ALTER TABLE {nazev_tabulky}  
ALTER COLUMN {nazev_sloupce}
TYPE VARCHAR (255);
\end{lstlisting}

Další fází modifikace stávající databáze bylo již samotné nahrazení identifikátoru \textbf{perun\_id} za \textbf{einfra id} v tabulkách \textbf{person}, \textbf{user\_settings}, \textbf{resource}, \textbf{acl\_identity} a \textbf{abstract\_reservation\_request}. V rámci zmíněné modifikace byly doplněny i dodatečné údaje o uživateli v držené tabulkách \textbf{person} a \textbf{user\_settings}. Jde o údaje o uživatelově jménu, emailové adrese, organizaci, preferovaném jazyku a časové zóně, které byly dříve získávány pomocí webové služby Perun. Tato služba se v nové implementaci již nepoužívá a proto musely být představené údaje pro všechny stávající uživatele v systému \texttt{meetings.cesnet.cz} uloženy do databáze. Výhodou je, že tabulky obsahují požadované sloupce, protože se s zmíněnými uživatelskými údaji již pracovalo při správě anonymních uživatelů. 
\par

Algoritmus, který popisované modifikace provedl jsem implementoval v jazyce Java a obsahuje následující kroky pro každou z tabulek vyžadující změnu. Počáteční fází bylo získání všech záznamů s uživatelskými identifikátory \textbf{perun\_id} z dotyčné tabulky. V dalším kroku byl pro každý záznam s identifikátorem \textbf{perun\_id} vytvořen GET požadavek na rozhraní webové služby Perun, které je přítomno na URL adrese \texttt{https://shongo-auth.cesnet.cz/perun/users/\{perun\_id\}}. Uvedený dotaz na API služby Perun vrátí požadované informace o uživateli ve formátu JSON s následující strukturou.
\begin{lstlisting}[language=json,firstnumber=1]
{
    "id": {id},
    "first_name": "{first_name}",
    "last_name": "{last_name}",
    "principal_names": [
        "{name@mail.cz}",
    ],
    "language": "{language}",
    "organization": "{organization}",
    "phone": "{phone}",
    "mail": "{name@mail.cz}",
    "display_name": "{display_name}",
    "sources": [
        {
            "id": {id},
            "name": "{name}",
            "loa": {loa},
            "login": "{login}"
        }
    ],
    "member_id": {member_id},
    "member_status": "{member_status}",
    "_links": {
        "self": {
            "href": "{href}"
        }
    }
}
\end{lstlisting}

Význam jednotlivých atributů odpovědi je popsán na dotaz volání služby Perun je dostupný v příloze \hyperref[table:perunws:user]{A.1}. 
\par
Dalším krokem algoritmu bylo získání identifikátoru \textbf{einfra id} z získané odpovědi. Tento identifikátor se nachází v poli \textbf{sources}, představující poskytovatele identit, konkrétně v atributu \textbf{sources.login}. Jelikož nás zajímá pouze identifikátor v rámci Proxy IdP infrastruktury CESNET, musel algoritmus najít poskytovatele identit, který má v atributu \textbf{sources.name} řetězec \textbf{https://login.cesnet.cz/idp/}. Jestli algoritmus našel zmíněného poskytovatele identit, v atributu \textbf{sources.login} se nachází \textbf{perun id} hledaného uživatele. Tento údaj byl poznamenán do nově vytvořeného objektu, představujícího uživatele. Do toho objektu byly dále uloženy informace o uživatelově jménu, emailové adrese, organizaci, preferovaném jazyku a časové zóně. 

\par

V závěrečné fázi algoritmu je pro každého takto vytvořeného objektu uživatele v aktuálně zpracovávané tabulce nahrazeno \textbf{perun\_id} za \textbf{einfra id}. Jak již bylo zmíněno, speciálním případem jsou tabulky \textbf{person} a \textbf{user\_settings}, kde došlo navíc k vložení chybějících uživatelských údajů. Následující obrázek \hyperref[fig:erd-person-user-settings]{4.1} ilustruje schéma tabulek \textbf{person} a \textbf{user\_settings}. 

\begin{figure}[H]
\label{fig:erd-person-user-settings}
\caption{Struktura tabulek \textbf{person} a \textbf{user\_settings}}
\centering
\includegraphics[width=12.8cm]{pics/erd-person-user_settings} 
\end{figure}
\par 

Zmíněnou modifikaci tabulky \textbf{person} pro každého uživatele ilustruje následující SQL dotaz.

\begin{lstlisting}[
           language=SQL,
           showspaces=false,
           basicstyle=\footnotesize\ttfamily,
           commentstyle=\color{gray},
           backgroundcolor=\color{background},
           frame=lines
        ]
UPDATE person SET 
user_id = {einfra_id},
email = {email},
name = {name},
organization = {organization},
WHERE id = {id};
\end{lstlisting}

Obdobně při modifikaci tabulky \textbf{user\_settings} byl použit následující SQL dotaz. 

\begin{lstlisting}[
           language=SQL,
           showspaces=false,
           basicstyle=\footnotesize\ttfamily,
           commentstyle=\color{gray},
           backgroundcolor=\color{background},
           frame=lines
        ]
UPDATE user_settings SET 
user_id = {einfra_id},
home_time_zone = {time_zone},
current_time_zone = {current_time_zone},
locale = {locale},
WHERE id = {id};
\end{lstlisting}

Je důležité poznamenat, že při zpracování tabulky \textbf{acl\_identity} byl postup odlišný. Uvedená tabulka má sloupce \textbf{id}, \textbf{principal\_id} a \textbf{type} ve kterých si drží své záznamy. Údaj ve sloupci \textbf{type} může nabývat hodnoty \textbf{USER} nebo \textbf{GROUP}. Jestliže nabývá hodnoty \textbf{USER}, potom je ve sloupci \textbf{principal\_id} obsažen identifikátor uživatele \textbf{perun\_id}. Pokud sloupec \textbf{type} obsahuje hodnotu \textbf{GROUP}, potom se ve sloupci \textbf{principal\_id} nachází identifikátor skupiny. Tato informace musela být zohledněna při nahrazování identifikátoru \textbf{principal\_id}. 
\par 
Pokud hodnota ve sloupci \textbf{principal\_id} představovala \textbf{perun\_id}, postupoval algoritmus jako v předchozím popisovaném případě a byl nahrazen za \textbf{einfra id}. Jestliže hodnota ve sloupci \textbf{principal\_id} představovala identifikátor skupiny, tento identifikátor musel být nahrazen za URN \footnote{URN RFC, \url{https://tools.ietf.org/html/rfc2141}} dotyčné skupiny. Uvedené URN (Uniform Resource Name) je textový řetězec, který se používá k jednoznačné identifikaci zdroje. Zmíněné nahrazení za URN bylo provedeno také pomocí webové služby Perun, ale s použitím rozhraní k získání dodatečných informací o požadované skupině. Volání představeného rozhraní bylo provedeno pomocí GET požadavku na URL adresu  \texttt{https://shongo-auth.cesnet.cz/perun/groups/\{group\_id\}}. Uvedený GET požadavek vrátil data o skupině v následující JSON struktuře.   
\begin{lstlisting}[language=json,firstnumber=1]
{
    "id": {id},
    "name": "{name}",
    "unique_name": "{unique_name}",
    "type": "{type}",
    "description": "{discription}",
    "admins": ["admin"],
    "_links": {
        "self": {
            "href": "{href}"
        },
        "users": {
            "href": "{href}"
        },
        "admins": {
            "href": "{href}"
        }
    }
}
\end{lstlisting}
Význam jednotlivých atributů v odpovědi služby Perun je popsán v příloze \hyperref[table:perunws:group]{A.2}. 
\par
Z uvedené struktury JSON byl použit atribut \textbf{unique\_name}, který reprezentuje požadované URN skupiny. Za tuto hodnotu URN skupiny byla potom hodnota v sloupci \textbf{principal\_id} pomocí SQL příkazu nahrazena. 

\subsection{Adaptace systému meetings.cesnet.cz na novou autentizační a autorizační vrstvu}
\label{ShongoImpl-web}
V této sekci se věnuji tématu, jakým způsobem musel být rezervační systém \texttt{meetings.cesnet.cz} adaptován na provedené změny v rámci implementace nové autentizační a autorizační vrstvy. 
\par

Prvním krokem byla změna URL adresy výchozího bodu autentizace, \textbf{UserInfo endpointu} a \textbf{Token endpointu}. Důvodem představené změny bylo, aby zmíněné adresy, jejichž využití vyžaduje autentizaci pomocí protokolu OpenID Connect, odpovídaly nastavení definovaném v rámci konfigurace Apache. \par


Výchozí část URL adresy, která slouží jako prefix URL adresy zajištující autentizaci a autorizaci je definována v souboru \texttt{cz/cesnet/shongo/client/web/ClientWebConfiguration.java}. Zmíněnou hodnotu, která je v stávající implementaci nastavena na URL adresu \texttt{/authn/oic} vrací metoda \texttt{getAuthenticationServerUrl}. Představenou část URL adresy jsem pozměnil na hodnotu \texttt{/oidc-authn/oic}, tak aby odpovídala URL adresám, které jsou chráněny OpenID Connect autentizací. Ostatní parametry URL adresy nebyly změněny. \par


\chapter{Závěr}
\printbibliography[title={Literatura}]

\appendix %% Start the appendices.
\chapter{Sémantika odpovědí webové služby Perun}
\section{Získání uživatele}
Následující údaje o uživateli jsou získány pomocí HTTP GET požadavku na URL adresu \texttt{https://shongo-auth.cesnet.cz/perun/users/\{perun\_id\}}
\label{table:perunws:user}
\begin{table}[h]
\begin{tabular}{|l|l|}
\hline
\texttt{id (int)}                   & \parbox[t]{6cm}{identifikátor uživatele v rámci Perun služby } \\ \hline
\texttt{first\_name (String)}       & křestní jméno uživatele \\ \hline
\texttt{last\_name (String)}        & příjmení uživatele \\ \hline
\texttt{principal\_names (Array)}   & \parbox[t]{6cm}{pole obsahující všechny eduPersonPrincipalName uživatele} \\ \hline
\texttt{language (String)}          & preferovaný jazyk uživatele \\ \hline
\texttt{organization (String)}      & výchozí organizace uživatele \\ \hline
\texttt{phone (String)}             & telefonní číslo uživatele \\ \hline
\texttt{mail (String)}              & emailová adresa uživatele \\ \hline
\texttt{display\_name (String)}     & úplné jméno uživatele \\ \hline
\texttt{sources (Array)}            & \parbox[t]{6cm}{pole poskytovatelů identit, v kterých je uživatel registrován} \\ \hline
\texttt{sources.id (int)}           & \parbox[t]{6cm}{identifikátor poskytovatele identit} \\ \hline
\texttt{sources.name (String)}      & název poskytovatele identit \\ \hline
\texttt{sources.loa (int)}          & \parbox[t]{6cm}{úroveň zabezpečení poskytovatele identit}  \\ \hline
\texttt{sources.login (String)}     & \parbox[t]{6cm}{identifikátor uživatele u daného poskytovatele identit} \\ \hline
\texttt{member\_id (int)}           & idetifikátor ve virtuální organizaci \texttt{meetings.cesnet.cz} \\ \hline
\texttt{member\_status (String)}    & zeptat se Miloše \\ \hline
\texttt{\_links.self.href (String)} & \parbox[t]{6cm}{URL adresa aktuálního HTTP GET požadavku} \\ \hline


\end{tabular}
\end{table}
\section{Získání skupiny}
Následující údaje o skupině jsou získány pomocí HTTP GET požadavku na URL adresu \texttt{https://shongo-auth.cesnet.cz/perun/groups/\{group\_id\}}

\label{table:perunws:group}
\begin{table}[h]
\begin{tabular}{|l|l|}
\hline
\texttt{id (int)}                       & \parbox[t]{6cm}{identifikátor skupiny v rámci Perun služby}  \\ \hline
\texttt{name (String)}                  & název skupiny  \\ \hline
\texttt{unique\_name (String)}          & URN skupiny  \\ \hline
\texttt{type (String)}                  &  typ skupiny \\ \hline
\texttt{description (String)}           & popis skupiny  \\ \hline
\texttt{admins (Array)}                 & pole administrátorů skupiny  \\ \hline
\texttt{\_links.self.href (String)}     & \parbox[t]{6cm}{URL adresa aktuálního HTTP GET požadavku}  \\ \hline
\texttt{\_links.users.href (String)}    & \parbox[t]{6cm}{URL adresa HTTP požadavku na získání uživatelů dané skupiny}  \\ \hline
\texttt{\_links.admins.href (String)}   & \parbox[t]{6cm}{URL adresa HTTP požadavku na získání administrátorů dané skupiny}  \\ \hline


\end{tabular}
\end{table}
\end{document}
