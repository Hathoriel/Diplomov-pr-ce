%%%%%%%%%%%%%%%%%%%%%%%%%%%%%%%%%%%%%%%%%%%%%%%%%%%%%%%%%%%%%%%%%%%%
%% I, the copyright holder of this work, release this work into the
%% public domain. This applies worldwide. In some countries this may
%% not be legally possible; if so: I grant anyone the right to use
%% this work for any purpose, without any conditions, unless such
%% conditions are required by law.
%%%%%%%%%%%%%%%%%%%%%%%%%%%%%%%%%%%%%%%%%%%%%%%%%%%%%%%%%%%%%%%%%%%%

\documentclass[
  printed, %% This option enables the default options for the
           %% digital version of a document. Replace with `printed`
           %% to enable the default options for the printed version
           %% of a document.
  twoside, %% This option enables double-sided typesetting. Use at
           %% least 120 g/m² paper to prevent show-through. Replace
           %% with `oneside` to use one-sided typesetting; use only
           %% if you don’t have access to a double-sided printer,
           %% or if one-sided typesetting is a formal requirement
           %% at your faculty.
  table,   %% This option causes the coloring of tables. Replace
           %% with `notable` to restore plain LaTeX tables.
  nolof,     %% This option prints the List of Figures. Replace with
           %% `nolof` to hide the List of Figures.
  nolot,     %% This option prints the List of Tables. Replace with
           %% `nolot` to hide the List of Tables.
  %% More options are listed in the user guide at
  %% <http://mirrors.ctan.org/macros/latex/contrib/fithesis/guide/mu/fi.pdf>.
]{fithesis3}
%% The following section sets up the locales used in the thesis.
\usepackage[resetfonts]{cmap} %% We need to load the T2A font encoding
\usepackage[T1,T2A]{fontenc}  %% to use the Cyrillic fonts with Russian texts.
\usepackage[
  main=czech, %% By using `czech` or `slovak` as the main locale
                %% instead of `english`, you can typeset the thesis
                %% in either Czech or Slovak, respectively.
  english, german, russian, czech, slovak %% The additional keys allow
]{babel}        %% foreign texts to be typeset as follows:
%%
%%   \begin{otherlanguage}{german}  ... \end{otherlanguage}
%%   \begin{otherlanguage}{russian} ... \end{otherlanguage}
%%   \begin{otherlanguage}{czech}   ... \end{otherlanguage}
%%   \begin{otherlanguage}{slovak}  ... \end{otherlanguage}
%%
%% For non-Latin scripts, it may be necessary to load additional
%% fonts:
\usepackage{paratype}
\def\textrussian#1{{\usefont{T2A}{PTSerif-TLF}{m}{rm}#1}}
%%
%% The following section sets up the metadata of the thesis.
\thesissetup{
    date          = \the\year/\the\month/\the\day,
    university    = mu,
    faculty       = fi,
    type          = mgr,
    author        = Bc. Lukáš Kotol,
    gender        = m,
    advisor       = {RNDr. Miloš Liška, Ph.D.},
    title         = {Autentizační a autorizační infrastruktura pro videokonferenční prostředí},
    TeXtitle      = {Autentizační a autorizační infrastruktura pro videokonferenční prostředí},
    keywords      = {keyword1, keyword2, ...},
    TeXkeywords   = {keyword1, keyword2, \ldots},
    abstract      = {This is the abstract of my thesis, which can

                     span multiple paragraphs.},
    thanks        = {These are the acknowledgements for my thesis, which can

                     span multiple paragraphs.},
    bib           = example.bib,
}
\usepackage{makeidx}      %% The `makeidx` package contains
\makeindex                %% helper commands for index typesetting.
%% These additional packages are used within the document:
\usepackage{paralist} %% Compact list environments
\usepackage{amsmath}  %% Mathematics
\usepackage{amsthm}
\usepackage{amsfonts}
\usepackage{url}      %% Hyperlinks
\usepackage{markdown} %% Lightweight markup
\usepackage{listings} %% Source code highlighting
\usepackage{hyperref}
\lstset{
  basicstyle      = \ttfamily,%
  identifierstyle = \color{black},%
  keywordstyle    = \color{blue},%
  keywordstyle    = {[2]\color{cyan}},%
  keywordstyle    = {[3]\color{olive}},%
  stringstyle     = \color{teal},%
  commentstyle    = \itshape\color{magenta}}
\usepackage{floatrow} %% Putting captions above tables
\floatsetup[table]{capposition=top}
\begin{document}

\chapter{Úvod}
Komunikace prostřednictvím videopřenosů a multimédií hraje v současném světě informačních technologií důležitou roli. Bez těchto moderních komunikačních prostředků si v dnešní době jen těžko dovedeme představit spolupráci v kolektivu, který tvoří osoby nacházející se na různých místech naší planety. Klíčovou vlastností zmíněných videokonferenčních systémů je dostupnost napříč uživatelskými organizacemi, jejichž spolupráci mají usnadňovat. Aby bylo možné propojení rozdílných organizací, je důležité mít videokonferenční platformu s vybudovanou autentizační a autorizační infrastrukturou, která tyto organizace dokáže integrovat. Pokud je autentizační infrastruktura navíc sdílena mezi další poskytovatele služeb, benefity představované platformy organizací a provozovatelů služeb, tzv. federace identit jsou nemalé. Jedná se například o fakt, že poskytovatelé služeb jsou osvobozeni od správy uživatelských údajů a hesel. Správce služby také definuje, jaký typ uživatele smí k jeho službě přistupovat, bez toho aby tyto politiky musel implementovat. Další výhodou federace identit je skutečnost, že uživateli stačí jen jedno přihlašovací jméno a heslo k přístupu ke všem poskytovaným službám. Uživatel přihlašovací údaje navíc vždy zadává na důvěrně známé webové stránce své organizace. V neposlední řadě, v případě kompromitace  přihlašovacích údajů si uživatel může v jednom kroku vygenerovat nové pro celou federaci identit. 

Cílem této diplomové práce je navrhnout a implementovat novou autentizační a autorizační infrastrukturu pro současné video a webkonferenční prostředí, které spravuje sdružení CESNET\footnote{CESNET, \url{www.cesnet.cz}}. Vytvořená autentizační a autorizační infrastruktura je založena na technologii OpenID Connect\footnote{OpenID Connect, \url{www.openid.net/connect/}}  a je integrována do stávající Proxy IdP v infrastruktuře CESNETu. Na novou autentizační a autorizační infrastrukturu jsou navázány služby Shongo \cite{shongoapi} a Adobe Connect\footnote{Adobe Connect, \url{www.adobe.com/products/adobeconnect.html}}, které také provozuje sdružení CESNET. Zmíněné videokonferenční služby mohou využívat uživatelé z organizací, které patří do České akademické federace identit eduID.cz \footnote{eduID.cz, \url{www.eduid.cz}} a její mezinárodní obdoby eduGAIN \footnote{eduGAIN, \url{www.edugain.org}}. Služba Shongo zprovozněná na portálu meetings.cesnet.cz slouží uživatelům k rezervaci výpočetních zdrojů a virtuálních místností,v kterých se uskutečňují následné videokonferenční spojení. Komerční platforma Adobe connect, provozovaná na URL connect.cesnet.cz\footnote{Adobe Connect CESNET Webkonference, \url{www.connect.cesnet.cz}}, umožňuje autentizovaným uživatelům uskutečňovat videokonferenční a webkonferenční schůzky v předem rezervovaných virtuálních místnostech. Zmíněná platforma Adobe Connect navíc podporuje textovou diskuzi, sdílení obsahu dokumentů, kreslící tabule a pracovní plochy počítače v prostředí prohlížeče. 

Výsledkem mé práce je plně funkční autentizační a autorizační infrastruktura zprovozněná v produkčním prostředí webkonferenční platformy sdružení CESNET. Vytvořenou autentizační a autorizační infrastrukturu jsem navrhnul a implementoval na serveru shongo-auth.cesnet.cz pomocí technologie OpenID Connect. Implementovaná infrastruktura je integrována do Proxy IdP provozované sdružením CESNET tak, aby autentizovala a autorizovala uživatele, využívající služeb portálů meetings.cesnet.cz a connect.cesnet.cz. Nová implementace oproti původní rozšiřuje způsoby přihlášení o možnosti použít účty společností Google, Facebook a dalších mezinárodních akademických organizací integrovaných do federace identit eduGAIN. Další předností nové implementace je vynechání autorizačního procesu pomocí externí webové služby Perun\footnote{Perun, \url{www.perun-aai.org}}, která poskytovala základní údaje o autentizovaném uživateli. Nyní jsou údaje přihlášeného uživatele zprostředkovány v rámci autentizačního procesu díky technologii OpenID Connect. Novou implementací došlo tedy k významnému zjednodušení složitosti autentizační a autorizační infrastruktury. \par

Text této diplomové práce je členěn následujícím způsobem. První část se věnuje úvodu. Po úvodu následuje druhá kapitola, ve které popisuji návrh infrastruktury autorizačního a autentizačního systému. V třetí kapitole se věnuji použitým technologiím, které jsem využil v rámci implementace diplomové práce. V následující čtvrté kapitole se zabývám popisem implementace. Poslední pátá kapitola shrnuje výsledky mé práce v závěru.  
\section{Slovník pojmů}

\chapter{Návrh autorizační a autentizační infrastruktury}
V této kapitole se budu věnovat návrhu autorizační a autentizační infrastruktury pro systémy Adobe Connect a Shongo. V následující sekci nejprve popisuji stávající stav autentizační a autorizační infrastruktury. Dále se v další sekci věnuji analýze požadavků na novou infrastrukturu. V poslední sekci této kapitoly se zabývám samotným návrhem nové autentizační a autorizační infrastruktury, který plyne z provedené analýzy stávajícího systému.    

\section{Popis stávající autentizační a autorizační infrastruktury}
Stávající autentizace a autorizace webkonferenční infrastruktury Shongo a Adobe Connect je založena na systémech Shibboleth \footnote{Shibboleth, \url{www.shibboleth.net}} a Perun \footnote{Perun, \url{www.perun-aai.org}}. Shibboleth je open-source software, který slouží jako autentizační vrstva s jednotným přihlašováním mezi několika systémy. Perun je centralizovaná služba určená pro správu uživatelských údajů. V dosavadní webkonferenční infrastruktuře, byla služba Perun využívána k získání dodatečných informací o autentizovaném uživateli. \par

Autentizační a autorizační infrastruktura využívající systémy Shibboleth a Perun je implementována na serveru \texttt{shongo-auth.cesnet.cz}. Uvedený server tedy odbavuje požadavky uživatelů na přihlášení do webkonferenčních systémů Shongo a Adobe Connect. Po úspěšné autentizaci je uživatel přesměrován na požadované portály. Abychom pochopily změny provedené ve stávající autentizační a autorizační infrastruktuře, je vhodné si detailně představit, jakým způsobem se zpracovávají požadavky na autentizaci uživatelů při přihlašování do zmíněných webkonferenčních systémů.  

\subsection{Infrastruktura Adobe Connect}
Jak už bylo zmíněno, webkonferenční platforma Adobe Connect využívá k autentizaci a
autorizaci systém Shibboleth. V rámci procesu autentizace si Adobe Connect udržuje interní databázi, ve které si eviduje uživatele. V
uvedené databázi si uchovává informace o uživatelích,
které získá po úspěšné autentizaci uživatele. Pokud záznam o uživateli
v databázi neexistuje, je vytvořen. Jelikož Adobe Connect využívá jen
identifikaci, jméno, příjmení a mail autentizovaného uživatele, není
potřeba využívat službu Perun. V dalším odstavci následuje detailní popis autentizační a autorizační infrastruktury systému Adobe Connect. \par

Zmíněná autentizační a autorizační infrastruktura je implementována v adresáři \texttt{/data/app/prod/aclogin-ng/} na který se budu v aktuální sekci odkazovat. Konfigurace představované implementace se nachází v podaresáři \texttt{config/} a samotná implementace autentizační a autorizační infrastruktury v podaresáři \texttt{lib/AcLogin/}. \par

Postup při přihlašování do webkonferenčního systému Adobe Connect provozovaného na URL adrese \texttt{www.connect.cesnet.cz} je následující. Pokud se chce uživatel do zmíněného systému přihlásit, klikne na přihlašovací odkaz, který ho přesměruje na URL adresu \texttt{https://shongo-auth.cesnet.cz/aclogin-ng/instance/connect}. Jelikož se jedná o adresu která je chráněna autentizačním systémem Shibboleth, uživatel je před samotným přístupem na uvedenou adresu přesměrován na rozcestník federace identit. Na zmíněném rozcestníku federace identit, který se nachází na URL \texttt{https://ds.eduid.cz/wayf.php} si uživatel vybere poskytovatele identity, pomocí kterého se přihlásí. Po přihlášení u vybraného poskytovatele identity je uživatel  přesměrován zpět na požadovanou URL \texttt{https://shongo-auth.cesnet.cz/aclogin-ng/instance/connect} a začne zpracování dat získaných po autentizaci. Jedná se o informace o autentizovaném uživateli, které byla nastaveny do proměnných prostředí PHP serveru, nacházející se v poli \texttt{\$\_SERVER}. Konkrétně jde o tyto indexy zmíněného pole: 

\begin{itemize}
    \item \textbf{eppn} představuje unikátní uživatelský identifikátor v rámci poskytovatele identit, 
    \item \textbf{mail} je emailová adresa uživatele, 
    \item \textbf{givenName} odkazuje na křestní jméno uživatele,
    \item \textbf{sn} představuje příjmení autentizovaného uživatele.
\end{itemize}
\label{item:adobe-connect}
Po nastavení proměnných prostředí se spustí hlavní metoda \texttt{\_run} PHP skriptu \texttt{Application.php} umístěném v adresáři \texttt{lib/AcLogin/}. Prvním krokem je v uvedené metodě inicializace konfigurace. Po té následuje vytvoření objektu představujícího uživatele pomocí třídy \texttt{AcLogin\_RemoteUser} z popisovaných proměnných prostředí. Dalším krokem je inicializace API objektu, usnadňující manipulaci s interní databází uživatelů systému Adobe Connect. Předtím než začne vytvořený objekt databázové rozhraní Adobe Connect využívat, provede se pomocí tohoto objektu přihlášení administrátora do databázového systému. \par 

Další fází v představovaném PHP skriptu je využití popisovaného Adobe Connect API objektu ke kontrole, zda uživatel, který se do systému přihlašuje již v interní databázi existuje. Představená kontrola existence uživatele se provede vůči jeho unikátnímu identifikátoru \textbf{eppn}. Pokud uživatel v databázi Adobe Connect neexistuje, je vytvořen a uložen do databáze. Jestliže uživatel v databázi existuje a je povolena dodatečná aktualizace uživatelských údajů při přihlášení, aktualizují se uživatelovi údaje v databázi Adobe Connect. \par

Po té následuje část zdrojového kódu, v které je provedeno samotné přihlášení uživatele uživatele pomocí unikátního identifikátoru \textbf{eppn} a generovaného hesla. Zmíněné heslo se generuje pomocí hašovací funkce MD5\footnote{Algoritmus hašovací funkce MD5, \url{www.tools.ietf.org/html/rfc1321}}, které se na vstup předá řetězec složený z \textbf{eppn} uživatele a náhodného neměnného řetězce, tzv. soli. Následuje kontrola úspěšnosti přihlášení a vytvoření řetězce sezení, tzv. \textbf{Session String}. Představený \textbf{Session String} jednoznačně identifikuje provedenou autentizaci. Pokud přihlášení proběhlo v pořádku, autentizovaný uživatel je s URL parametrem \textbf{Session String} přesměrován na svoji úvodní stránku webkonferečního systému Adobe Connect.      

\subsection{Infrastruktura Shongo}
Podobně jako autentizační infrastruktura systému Adobe Connect je i autentizační infrastruktura rezervačního systému Shongo implementována na serveru \texttt{shongo-auth.cesnet.cz}. Systém Shongo využívá při přihlašování uživatelů jak Shibboleth, tak webovou službu Perun. Je důležité poznamenat, že autentizační infrastruktura využívá moduly \textbf{shongo-authn-server} a \textbf{zf2-openid-connect-server-module} k zpracování autentizačních požadavků. Uvedené moduly se nacházejí v adresáři \texttt{/data/app/prod/shongo-authn-server/}. Ačkoliv se jedná o implementaci OpenID Connect na straně poskytovatele služeb, nejsou moduly napojeny na autorizační server OpenID Connect, ale pouze na systém Shibboleth v autentizační infrastruktuře CESNETu. \par

Postup při přihlašování uživatele do systému Shongo je následující. Pokud se chce uživatel autentizovat, klikne na přihlašovací odkaz, který ho přesměruje na URL adresu \texttt{shongo-auth.cesnet.cz/authn/oic/authorize}. Požadavek na přístup na uvedenou URL adresu obsahuje další parametry nezbytné pro dokončení autentizace:
\begin{itemize}
    \item \textbf{client\_id} má hodnotu \texttt{meetings.cesnet.cz} a představuje identifikace klienta u autorizačního serveru,
    \item \textbf{redirect\_url} URL adresa na kterou prohlížeč přesměrován po autentizaci, má hodnotu \texttt{https://meetings.cesnet.cz/login} 
    \item \textbf{state} hodnota sloužící k udržení stavu mezi přihlašovacím požadavkem a odpovědí na autentizaci 
    \item \textbf{scope} obsahuje hodnotu \textbf{openid}, která značí že se jedná požadavek protokolu OpenID Connect. 
    \item \textbf{response\_type} definuje způsob průběhu získávání tokenů, v popisovaném požadavku má hodnotu \textbf{code}, což značí že se jedná o tzv. Authorization Code Flow \footnote{OpenID Connect Authorization Code Flow, \url{https://openid.net/specs/openid-connect-core-1_0.html\#CodeFlowAuth}}
    \item \textbf{prompt} doplňková informace autorizačnímu serveru, požadující zobrazení autentizačního formuláře koncovému uživateli, i pokud je přihlášen. 
\end{itemize}


Zmíněná URL adresa je namapována na metodu \texttt{authorizeAction} třídy \texttt{AuthorizeController}, která se nachází v modulu \textbf{zf2-openid-connect-server-module} jakožto i následující představované třídy. V uvedené metodě \texttt{authorizeAction} začíná běh autentizačního algoritmu. \par 

Dále je v metodě \texttt{authorizeAction} provedena kontrola, zda je uživatel již přihlášen. Pokud tato skutečnost nastala, předá se řízení algoritmu metodě \texttt{responseAction} nacházející se v aktuální třídě \texttt{AuthorizeController}. Uvedená metoda zpracuje data udržované v kontextu o přihlášeném uživateli. Potom jsou generovány na straně serveru \texttt{shongo-authn.cesnet.cz} textové řetězce, tzv. \textbf{session} a \textbf{authorize code}. Uvedený řetězec \textbf{session} jednoznačně identifikuje vytvořené sezení. \textbf{Authorize code} je textový řetězec, který po autentizaci obdrží klient od autorizačního serveru. Zmíněný \textbf{Authorize code} slouží k jednorázové výměně za tzv. \textbf{ID Token} nebo \textbf{Access Token}. Dalšímu použití uvedených žetonů se detailněji věnuji v rámci popisu technologií \hyperref[sec:oauth]{OAuth} a \hyperref[sec:oidc]{OpenID Connect}. S retězcem \textbf{Authorize Code} je uživatel přesměrován na úvodní stránku přihlášeného uživatele portálu Shongo. Metodě \texttt{responseAction} se dále věnuji v \hyperref[sec:responseAction]{odstavci níže}. \par 

Jestliže uživatel autentizovaný ještě není, je přesměrován na URL adresu \texttt{shongo-auth.cesnet.cz/authn/oic/authn/shibboleth}. Jelikož je v nastavení Apache uvedená URL adresa vedena jako chráněná systémem Shibboleth, je uživatel přesměrován na již popisovaný rozcestník federace identit. Po přihlášení u vybraného poskytovatele identit se nastaví základní informace o autentizovaném uživateli do proměnných prostředí PHP serveru. Jedná se o proměnné v poli \texttt{\$\_SERVER} v indexech \textbf{epp}, \textbf{mail}, \textbf{givenName} a \textbf{sn}. Jde tedy o \hyperref[item:adobe-connect]{stejná informace získané o uživateli jako  v případě autentizace do systému Adobe Connect}. Dále je provedeno přesměrování zpět na URL adresu \texttt{shongo-auth.cesnet.cz/authn/oic/authn/shibboleth}, kde se již spustí metoda \texttt{authenticateAction} třídy \texttt{ShibbolethController}. \par

V představené metodě \texttt{authenticateAction} je nastaven objekt \texttt{ShongoAuthn\textbackslash User\textbackslash User} pomocí získaných údajů z proměnných prostředí PHP serveru. Uvedený objekt představuje autentizovaného uživatele a drží si všechny získané informace o daném uživateli. Dále je tento objekt uložen do kontextu proměnné \texttt{\$\_SESSION}. Hodnoty uložené v proměnné \texttt{\$\_SESSION} jsou uchovávané v paměti PHP serveru a jsou sdíleny mezi jednotlivými HTTP požadavky. Pokud všechny předchozí kroky proběhly úspěšně, je předáno řízení algoritmu metodě \texttt{responseAction} nacházející se v třídě \texttt{AuthorizeController}. \par

\label{sec:responseAction}
V metodě \texttt{responseAction} se spouští klíčová metoda \texttt{dispatch} třídy \texttt{Authorize} v které pokračuje algoritmus. Zmíněná metoda \texttt{dispatch} používá metodu \texttt{populateUser} třídy \texttt{PerunWS}, která implementuje rozhraní \texttt{DataConnectorInterface}. \par

Uvedená třída \texttt{PerunWS} se nachází v modulu \textbf{shongo-authn-server}. V metodě \texttt{populateUser}, kterou implementuje třída \texttt{PerunWS} se získávají dodatečné informace o autentizovaném uživateli pomocí API webové služby Perun. Nejprve algoritmus z \textbf{eppn} zjistí unikání identifikátor \textbf{perun\_id} v kontextu Perun služby. Potom následuje další dotaz na Perun službu, kterým se algoritmus dotazuje na dodatečné informace o autentizovaném uživateli pomocí \textbf{perun\_id}. Jde o tyto uživatelské informace:

\begin{itemize}
    \item \textbf{phone} je telefon uživatele,
    \item \textbf{organization} odkazuje na organizaci do které uživatel patří,
    \item \textbf{language} představuje preferovaný jazyk uživatele,
    \item \textbf{timezone} je časová zóna do které uživatel patří, 
    \item \textbf{principal\_names} představuje pole unikátních identifikátorů osoby v rámci federace dané federace. 
\end{itemize}
Dále jsou získány již dříve známé informace o křestním jménu, příjmení a emailové adrese uživatele. Uvedené informace o uživateli jsou uloženy do již popisovaného objektu třídy \texttt{ShongoAuthn\textbackslash User\textbackslash User}. \par

Po té se předá řízení algoritmu zpět do metody \texttt{dispatch} v třídě \texttt{Authorize}. V této metodě je dále z získaných uživatelských dat vytvořeno tzv. \textbf{session} a \textbf{Authorize code}. Uvedené \textbf{session} je vygenerovaný náhodný textový řetězec, reprezentující jedinečné spojení získané autentizací uživatele. Ostatní níže představené pojmy včetně \textbf{Authorize code} jsou definovány v rámci popisu technologií \hyperref[sec:oauth]{OAuth} a \hyperref[sec:oidc]{OpenID Connect}. Uvedené \textbf{session} a \textbf{Authorize code} jsou uloženy do interní databáze serveru \texttt{shongo-authn.cesnet.cz}. Tyto uložené údaje jsou později využity při validaci požadavku na získání \textbf{Access Tokenu} a dalších uživatelských informací. Nakonec je uživatel retězcem \textbf{Authorize Code} přesměrován na úvodní stránku přihlášeného uživatele portálu Shongo. \par

Autentizovanému \textbf{Clientu} Shongo je po té vydán  \textbf{Access Token} na základě požadavku s přiloženým \textbf{Authorize code}.   
Zdroj poskytující \textbf{Access Token}, tzv. \textbf{Token Endpoint} je přítomný na URL adrese \texttt{shongo-auth.cesnet.cz/authn/oic/token}. \par 

Obdobně jsou autentizovanému \textbf{Clientu} poskytnuty dodatečné uživatelské údaje na základě přiloženého \textbf{Authorize code}. Zdroj který popisované uživatelské údaje drží, tzv. \textbf{UserInfo Endpoint} je dostupný na URL adrese \texttt{shongo-auth.cesnet.cz/authn/oic/userinfo}.  \par 

Následující \hyperref[fig:shongoAuthnProcess]{diagram} znázorňuje přehlednou formou výše popsané procesy v stávající autentizační a autorizační infrastruktuře Shongo. 

\begin{figure}[H]
\caption{Schéma komunikace při zpracování autentizačního požadavku do systému Shongo}
\centering
\includegraphics[width=12.8cm]{pics/shongoAuthProcess} 
\label{fig:shongoAuthnProcess}
\end{figure}
\par 


\section{Analýza požadavků}
Z konzultací s vedoucím diplomové práce vyplynuly následující požadavky. Jak už bylo uvedeno v úvodu, hlavní podstatou této diplomové práce je vytvoření nové autentizační a autorizační infrastruktury pro portály \texttt{connect.cesnet.cz} a \texttt{meetings.cesnet.cz} provozované sdružením CESNET. Požadavkem bylo, aby tato nová autentizační a autorizační infrastruktura byla implementována v souladu se specifikací OpenID Connect  \footnote{OpenID Connect specifikace, \url{www.openid.net/specs/openid-connect-core-1_0.html}}. Nezbytnou součástí implemtace bylo, aby autentizační a autorizační infrastruktura byla navázána na Proxy IdP OpenID Connect, které provozuje sdružení CESNET\footnote{Proxy IdP OpenID Connect, \url{www.login.cesnet.cz/oidc/}}. Proxy IdP je komponenta mezi poskytovatelem služeb a identit, která umožňuje systému aby byla dostupná prostřednictvím různých poskytovatelů identit. \par

Dalším požadavkem bylo, aby implementace byla z důvodu centralizace autentizanční a autorizační infrastruktury zakomponována do stávajícího serveru \texttt{shongo-auth.cesnet.cz}. V neposlední řadě bylo požadováno odstranění nepotřebného kódu po nevyužitých voláních webové služby Perun. Informace o přihlášeném uživateli, které byly zprostředkovány voláním webové služby Perun musely být nově získány z autorizačního serveru poskytovatele OpenID Connect v infrastruktuře CESNETu.     


\section{Návrh nové infrastruktury}
Zde se budu věnovat otázce, jakým způsobem je koncipován návrh nové autentizační a autorizační infrastruktury pro systémy Adobe Connect a Shongo. Společným prvkem pro oba zmíněné systémy je fakt, že návrh nové autentizační a autorizační architektury navazuje na architekturu stávající. 

\subsection{Infrastruktura Adobe Connect}
Nová implementace bude využívat již existující autentizační a autorizační infrastrukturu a bude s ní úzce svázána. To znamená, že bude dodána implementace do stávajícího autentizačního serveru \texttt{shongo-authn.cesnet.cz}. Jak už bylo uvedeno výše, stávající autentizační a autorizační infrastruktura využívá systému Shibboleth, který bude v rámci této diplomové práce nahrazen systémem OpenID Connect. To znamená, že bude muset být systém OpenID Connect a moduly na straně autentizačního serveru \texttt{shongo-authn.cesnet.cz} do sebe vhodným způsobem integrovány. Tuto skutečnost dosáhneme řádnou instalací  autentizačního modulu OpenID Connect webovéo serveru Apache na autentizační server. Dále musíme zajistit, aby došlo k přemapování proměnných získaných po autentizaci uživatele. Implementaci zmíněného přemapování popisuji v kapitole \hyperref[ACImpl]{Implementace autentizačního a autorizačního modulu pro systém Adobe Connect}. 

\subsection{Infrastruktura Shongo}
Obdobně jako v případě systému Adobu Connect, bude nová implementace autorizace a autentizace uživatelů úzce propojena se stávající implementací, která se nachází na serveru \texttt{shongo-auth.cesnet.cz}. Analogicky bude muset být provedena instalace Apache modulu OpenID Connect a konfigurace v rámci nastavení webového serveru. \par
V další fázi bude na serveru \texttt{shongo-auth.cesnet.cz} následovat implementace procesu, který zajistí propojení s autorizačním serverem OpenID Connect. Uvedená implementace zaručí mapování nově získaných uživatelských dat do stávajícího mechanizmu autentizace a autorizace uživatele. \par
Dalším krokem je adaptace systému Shongo na provedené změny v rámci implementace na straně serveru \texttt{shongo-auth.cesnet.cz}. Tuto adaptaci představují dva klíčové body. Prvním je nutnost změny v databázovém schématu týkající se tabulek v kterých si systém Shongo drží identifikátory svých uživatelů. Jedná se o zvětšení velikosti sloupců držící uživatelské identifikátory, jejichž velikost není v rámci stávající implementace dostačující. Příčinou této změny je nahrazení identifikátoru uživatele \textbf{perun\_id} za \textbf{einfra id}, jehož délka by překračovala stávající velikost sloupce. Zmíněné identifikátory jednoznačně identifikují uživatele v rámci služby Perun, respektive v OpenID Connect infrastruktuře CESNETu. Tato změna vychází z nutnosti zbavit se závislosti na webové službě Perun. \par

Druhým bodem je nutnost uložení dodatečných informací o autentizovaném uživateli v rámci systému Shongo. Zmíněná data jsou ve stávající implementaci získávána při každém požadavku na zobrazení uživatelských informací pomocí služby Perun. Jelikož nová implementace nebude závislá na webové službě Perun, musí se představené uživatelské informace ukládat do databáze již v rámci procesu autentizace uživatele. Díky odstranění závislosti autentizační a autorizační infrastruktury na webové službě Perun dojde k nemalé snížení složitost zmíněného systému. Implementaci uvedených změn popisuji v kapitole \hyperref[ShongoImpl]{mplementace autentizačního a autorizačníhomodulu pro systém Shongo}.

\chapter{Použité technologie}
V této kapitole popisuji technologie, které jsem použil při implementaci mé diplomové práce. Popis doplňují informace o tom, v které konkrétní části implementace jsem danou technologii použil.
\section{OAuth}
\label{sec:oauth}
OAuth 2.0 \footnote{OAuth 2.0 RFC, \url{www.tools.ietf.org/html/rfc6749}} je autorizační framework, který umožňuje aplikacím třetí strany získat omezený přístup k HTTP službám. Tento framework je tedy určen pro bezpečné delegování přístupu. Zmíněný framework představuje autorizační vrstvu, která odděluje role klienta od držitele zdrojů. Specifikace OAuth 2.0 popisuje následující role: 
\begin{itemize}
    \item \textbf{Resource Owner} je entita, schopná dát souhlas s přístupem k požadovanému zdroji, většinou koncový uživatel,
    \item \textbf{Resource Server} představuje server, uchovávající chráněné zdroje, schopný odpovídat požadavkům přistupujícím k chráněným zdrojům,
    \item \textbf{Client} je aplikace, vytvářející požadavky na získání chráněných zdrojů se svolením role \textbf{Resource Owner},
    \item \textbf{Authorization Server} je server, vydávající \textbf{Access Token} pro roli \textbf{Client} po úspěšném ověření identity \textbf{Resource Owner} a získání oprávnění.
\end{itemize}
Uvedený pojem \textbf{Access Token} je pověření používané k získání přístupu k chráněným zdrojům. \textbf{Access Token} se obvykle vydává ve formátu \textbf{JWT (JSON Web Token)}, kterému se věnuji v následující sekci v rámci popisu technologie \hyperref[sec:oidc]{OpenID Connect}.  \par

Schéma komunikace v protokolu OAuth 2.0 popisuje následující obrázek \hyperref[fig:oauth]{3.1}. 

\begin{figure}[H]
\caption{Schéma komunikace v protokolu OAuth 2.0}
\centering
\includegraphics[width=12.8cm]{pics/diplomkaOauth} 
\label{fig:oauth}
\end{figure}
\par 

V prvním kroku (A) \textbf{Client} požaduje autorizaci po \textbf{Resource Owner}. V další fázi (B) \textbf{Client} získá pověření, představující oprávnění vlastníka k přístupu ke zdrojům. V následujícím kroku (C) \textbf{Client} požaduje \textbf{Access Token} po \textbf{Authorization Server}, kterému prezentuje udělená oprávnění. Další krok (D) představuje autentizaci a validaci uděleného oprávnění, na základě kterého je vydán \textbf{Access Token}. V předposlední fázi (E) \textbf{Client} přistupuje k \textbf{Resource Server} s \textbf{Access Tokenem} a požadavkem na získání chráněného zdroje. Po validaci \textbf{Access Tokenu} poskytne \textbf{Resource Server} požadované zdroje entitě \textbf{Client} (F). \par 

Autorizační protokol OAuth 2.0 není navržen pro autentizaci uživatelů a neposkytuje možnost získání dalších informací o přihlášeném uživateli. Z těchto důvodů byl v implementaci použit spolu s technologií OpenID Connect, která zmíněnou funkcionalitu definuje a umožňuje ji vhodně implementovat. 

\section{OpenID Connect}
\label{sec:oidc}
OpenID Connect 1.0 \footnote{Dokumentace OpenID Connect 1.0, \url{www.openid.net/specs/openid-connect-core-1_0.html}} je protokol, který umožňuje aplikaci verifikovat identitu uživatele na základě autentizace provedené autorizačním serverem. Jedná se o autentizační vrstvu nad frameworkem OAuth 2.0. Důležitou vlastností OpenID Connect je možnost získat základní informace o přihlášeném uživateli. \par

Hlavním rozšířením protokolu OpenID Connect oproti OAuth 2.0 umožňující autentizaci je definice nové datové struktury \textbf{ID Token}. Uvedený \textbf{ID Token} je bezpečnostní žeton, který obsahuje informace o autentizaci a případná další doplňková data koncového uživatele. Zmíněné informace nacházející se ve struktuře \textbf{ID Token} nazýváme v tomto kontextu \textbf{Claims}. Popisované \textbf{Claims} vždy nesou informace ve formátu dvojice název \textbf{Claim} a hodnota \textbf{Claim}. Představované \textbf{Claims}, které uchovávají informace o autentizované entitě, jsou ve struktuře \textbf{ID Token} zakódovány ve formátu \textbf{JWT} (JSON Web Token)\footnote{JSON Web Token RFC, \url{www.tools.ietf.org/html/rfc7519}}.    \par

\textbf{JWT} je textový řetězec reprezentující \textbf{Claims} jako JSON objekt\footnote{JavaScript Object Notation RFC, \url{https://tools.ietf.org/html/rfc7159}} zakódovaný pomocí Base64\footnote{Base64 kódování, \url{www.tools.ietf.org/html/rfc4648}} kódování a HMAC s SHA-256\footnote{HMAC s SHA-256, \url{ www.tools.ietf.org/html/rfc4868}}.  Popisovaný \textbf{JWT} je standardně rozdělen na části hlavičku a tělo, ze kterých se po aplikaci Base64 a HMAC s SHA-256 na tyto dvě části vytváří výsledný řetězec. Konkrétní mechanizmus způsobu vytváření \textbf{JWT} je detailně popsán v \href{http://www.tools.ietf.org/html/rfc7519}{odkazovaném RFC 7519} . \par

Protokol OpenID Connect dále definuje jaké \textbf{Claims} musí struktura \textbf{ID Token} obsahovat:

\begin{itemize}
    \item \textbf{iss} představuje identifikátor entity, která vytvořila \textbf{ID Token}, 
    \item \textbf{sub} označuje identifikátor autentizovaného uživatele, 
    \item \textbf{aud} je pole identifikátorů entit, pro které je vydaný \textbf{ID Token} určen,
    \item \textbf{exp} reprezentuje časový údaj, do kdy je vydaný \textbf{ID Token} platný
    \item \textbf{iat} označuje informaci, kdy byl \textbf{ID Token} vydán.
\end{itemize}
\par

Před samotným rozborem schématu komunikace v protokolu OpenID Connect je vhodné definovat následující pojmy. \textbf{OpenID Provider (OP)} představuje autorizační server schopný autentizovat koncového uživatele. Jednou z klíčových funkcí \textbf{OP} je možnost poskytnout \textbf{Relying Party} \textbf{Claims} o autentizační události a autentizovaném uživateli. Zmíněný pojem \textbf{Relying Party (RP)} označuje klientskou aplikaci, požadující autentizaci uživatele a \textbf{Claims} od \textbf{OP}. \par
Následující obrázek \hyperref[fig:oidc]{3.2} popisuje s použitím dříve definovaných pojmů schéma komunikace v protokolu OpenID Connect.

\begin{figure}[H]
\caption{Schéma komunikace v protokolu OpenID Connect 1.0}
\centering
\includegraphics[width=12.8cm]{pics/diplomkaOIDC} 
\label{fig:oidc}
\end{figure}
\par 

V první fázi (A) pošle \textbf{Relying Party} autentizační požadavek autorizačnímu serveru \textbf{OpenId Provider}. V dalším kroku (B) \textbf{OP} autentizuje koncového uživatele a provede autorizaci. V třetí fázi (C) \textbf{OP} odpoví autorizační server na požadavek informací o výsledku autentizace spolu s žetony \textbf{ID Token} a \textbf{Access Token}. V dalším kroku (D) \textbf{RP} může pomocí \textbf{Access Tokenu} poslat \textbf{OP} požadavek na získání uživatelských dat. Nakonec poslední fází (E) komunikačního schématu je odpověď \textbf{OP} s \textbf{Claims} obsahující uživatelská data, které si v předchozím kroku \textbf{RP} vyžádal. \par

V předchozím odstavci je zmíněna předposlední fáze \textbf{D} ve které \textbf{RP} po autentizaci posílá požadavek \textbf{OP} na získání dat o uživateli. Zdroj, který tyto informace uchovává se nazývá \textbf{UserInfo Endpoint}. Představovaný \textbf{UserInfo Endpoint} je chráněná URL adresa, která akceptuje GET a POST HTTP požadavky pouze s přiloženým validním \textbf{Access Tokenem} získaným při autentizaci. Formát odpovědi na tyto dotazy se nastavuje při registraci \textbf{OpenID Providera}.  \par

Jak už bylo zmíněno v sekci která se věnuje popisu stávající implementace systému Shongo, proces autentizace a autorizace je řízen pomocí tzv. Authorization Code flow. Tento způsob získání \textbf{ID Tokenu} a \textbf{Access Tokenu} vkládá do popisovaného komunikačního schématu v protokolu OpenID Connect další mezi krok. Místo zmíněných žetonů je v kroku (C) autorizačním serverem \textbf{OP} vrácen vygenerovaný textový řetězec tzv. \textbf{Authorize code}. Tento \textbf{Authorize code} je v následujícím kroku vyměněn za požadované žetony \textbf{ID Token} a \textbf{Access Token}. Podobně jako v případě \textbf{UserInfo Endpoint}, chráněný zdroj na straně autorizačního serveru \textbf{OP}, který poskytuje žetony na základě validace \textbf{Authorize code} se nazývá \textbf{Token Endpoint}.

\par
OpenID Connect je klíčová technologie, který byla využita k vytvoření autentizační a autorizační infrastruktury při vypracování této diplomové práce. Existuje velké množství certifikovaných implementací technologie OpenID Connect v mnoha různých programovacích jazycích\footnote{Certified OpenID Connect implementations, \url{www.openid.net/developers/certified/}}. V této diplomové práci jsem použil MITREid Connect\footnote{MITREid Connect, \url{www.mitreid-connect.github.io/}}, jelikož je již integrována v infrastruktuře CESNETu jako implementace OpenID Connect. Jedná se o webovou aplikaci v programovacím jazyce Java, která je založená na frameworku Spring Security\footnote{Spring Security framework documentation, \url{www.docs.spring.io/spring-security/site/docs/current/reference/htmlsingle/}}. 

\section{Apache HTTP server}
Apache HTTP server \cite{apache} (zkráceně Apache) je open-source multiplatformní software, sloužící jako webový server. Apache podporuje velké množství modulů, které rozšiřují základní funkce jádra webového serveru.  \par

Webový server Apache verze 2 jsem v autentizační a autorizační infrastruktuře využil jako základní prostředek pro zpracování uživatelských HTTP požadavků. Zejména jde o HTTP požadavky na přihlášení do videokonferenčních systémů Shongo a Adobe Connect. 
\par
K tomu aby mohl být server Apache integrován s OpenID Connect, musel být nainstalován modul \textbf{mod\_auth\_openidc}. Zmíněný modul umožňuje autentizovat a autorizovat klientskou aplikaci vůči autorizačnímu serveru OpenID Connect. Modul dále může sloužit i pro nastavení serveru, který funguje jako tzv. \textbf{Resource Server} v rámci protokolu OAuth 2.0. Konkrétní způsob konfigurace Apache spolu s OpenID Connect modulem popisuji v sekci \hyperref[apacheConfig]{Konfigurace Apache}.

\section{PHP}
PHP \cite{php5} (PHP: Hypertext Preprocessor) je široce používaný open-source objektově orientovaný skriptovací programovací jazyk, určený především pro tvorbu webových stránek. Zmíněný programovací je multiplatformní, kompatibilní s většinou běžně používaných serverů a podporuje připojení k různým databázovým systémům. 
\par

V představovaném programovacím jazyce PHP jsem na straně serveru \texttt{shongo-auth.cesnet.cz} implementoval zpracování Claims získaných z UserInfo Endpointu OpenID Connect po autentizaci uživatele. Architektura autentizačního serveru je vystavěna na aplikačním frameworku Zend. \footnote{Zend Framework, www.framework.zend.com}. Jedná se o open-source webový aplikační framework, s důrazem na vyvíjení jednoduchých webových aplikací. Zmíněná implementace je detailněji popsána v kapitolách Implementace autentizačního a autorizačního modulu pro \hyperref[ACImpl]{Adobe Connect} a \hyperref[ShongoImpl]{Shongo}. \par

\section{PostgreSQL}
PostgreSQL \cite{postgresql} je open-source multiplatformní objektově relační databázový systém napsaný v programovacím jazyce C. Tento databázový systém je založen na dotazovacím jazyku SQL (Structured Query Language), který je používám pro práci s daty. 
\par

V rámci této diplomové práce jsem pracoval s popisovaným databázovým systémem při modifikaci databázového schématu rezervačním systému Shongo. Dále jsem PostgreSQL využil při nahrazení primárního klíče uživatele z \textbf{perun\_id} na unikátní identifikátor uživatele v rámci OpenID Connect \textbf{sub}. Tuto změnu detailněji popisuji v sekci TODO. 

\section{Hibernate}
Hibernate ORM\footnote{Hibernate ORM framework , \url{www.hibernate.org/orm/}} (Object/Relational Mapping) je objektově relační framework napsaný v jazyce Java. Jedná se o nástroj jehož hlavní funkcí je mapování objektů v jazyce Java na entity v relační databázi.
\par
Tento framework je použit ve stávající implementaci rezervačního systému Shongo a využil jsem jej při modifikaci chování datové vrstvy tohoto systému.  


\chapter{Popis implementace}
V této kapitole seznamuji s detailním postupem při implementaci a instalaci nové autentizační a autorizační infrastruktury využívající protokol OpenID Connect. V kapitole \hyperref[apacheConfig]{4.1} popisuji jakým způsobem probíhala instalace knihoven sloužících pro zprovoznění modulu OpenID Connect pro webový server Apache. V dalších kapitolách \hyperref[ACImpl]{4.2} a \hyperref[ShongoImpl]{4.3} se věnuji implementaci autentizační a autorizační infrastruktury systému Adobe Connect, respektive systému Shongo.   


\section{Instalace a konfigurace Apache modulu pro OpenID Connect}
\label{apacheConfig}
\subsection{Instalace knihoven modulu mod\_auth\_openidc}
Příkazy popisované během instalace Apache modulu pro OpenID Connect byly v rámci diplomové práce prováděny na serveru \texttt{shongo-auth.cesnet.cz} s operačním systémem Debian GNU/Linux 7 (wheezy). 
\par 
Autentizační a autorizační modul, který slouží pro zprovoznění serveru jako tzv. \textbf{OpenID Connect Relying Party} se nazývá \textbf{mod\_auth\_openidc\footnote{OpenID Connect Relying Party pro Apache server \url{https://github.com/zmartzone/mod\_auth\_openidc}}}. Důležitým předpokladem pro instalaci představeného modulu je již dříve nainstalovaný server Apache. 
\par 
Prvním krokem byla instalace nezbytných knihoven \textbf{libjansson4} a \textbf{libhiredis0.13}, \textbf{libcjose0\_0.5} a \textbf{libapache2-mod-auth-openidc\_2.3.3-1}.

\subsection{Registrace klienta do Proxy IdP infrastruktury}

Pokud instalace uvedených knihoven proběhla úspěšně následovala registrace klienta \texttt{shongo-auth.cesnet.cz} do Proxy IdP infrastruktury sdružení CESNET. Zmíněná registrace probíhala ve webové aplikaci MITREid Connect na adrese \url{https://login.cesnet.cz/oidc/}. Po přihlášení do webové aplikace bylo v uživatelském rozhraní v levém menu v kolonce \textbf{Developer} vybráno \textbf{Self-service client registration}. Uvedená akce přesměrovala prohlížeč na stránku s tlačítkem \textbf{New Client} sloužící pro registraci nového klienta. Po kliknutí na zmíněné tlačítko aplikace přesměrovala prohlížeč na registrační formulář s konfigurací nového klienta. V záložce \textbf{Main} byla vyplněna kolonka \textbf{Redirect URI (s)} adresou \texttt{https://shongo-auth.cesnet.cz/oauth2callback} představující URL adresu pro přesměrování po autentizaci koncového uživatele. Dále byly vyplněny údaje v záložce \textbf{Access} v kolonce \textbf{Scope}. Zmíněné \textbf{Scopes} indikují údaje o uživateli, ke kterým bude chtít registrovaný klient po autentizaci přistupovat. V rámci registrace klienta byly vybrány následující \textbf{Scopes}.
\begin{itemize}
    \item \textbf{openid} představuje unikátní identifikátor uživatele v rámci Proxy IdP ifrastruktury, tzv. einfra id,
    \item \textbf{groupNames} obsahuje seznam skupin, kterých je uživatel členem,
    \item \textbf{address} je poštovní adresa uživatele,
    \item \textbf{phone} odkazuje na telefonní číslo uživatele,
    \item \textbf{profile} jde o osobní profil uživatele požadující \textbf{Claimy} \textbf{name}, \textbf{given\_name}, \textbf{middle\_name}, \textbf{family\_name}, \textbf{preferred\_username}, \textbf{zoneinfo} a \textbf{locale},
    \item \textbf{organization} představuje domovskou organizaci ke které uživatel patří,
    \item \textbf{eppns} značí seznam všech \textbf{eduPersonPrincipalName} uživatele, jde o jednoznačné identifikátory v rámci federace, mající formát uživatelské\_jméno@doména,  
    \item \textbf{email} označuje emailovou adresu uživatele.
\end{itemize}
Dále je důležité zmínit, že v kolonce \textbf{Grant Types} byl zvolen \textbf{authorization code}, který respektuje autentizační a autorizační architekturu registrovaného serveru \texttt{shongo-auth.cesnet.cz}.  \par

Po uložení zmíněné konfigurace byly aplikací náhodně vygenerovány tyto následující údaje. Jde o \textbf{Client ID}, které jednoznačně identifikuje registrovaného klienta. Dále byl vygenerován registrační token, tzv. \textbf{Registration Access Token}, který slouží pro pozdější zpřístupnění editace konfigurace vytvořeného klienta. Posledním údajem, který byl vygenerován je sdílené tajemství mezi klientskou aplikací a autorizačním serverem, tzv. \textbf{Client Secret}. Zmíněný \textbf{Client Secret} autentizuje požadavky klientské aplikace vůči autorizačnímu serveru. Vygenerované údaje byly dále využity v rámci konfigurace modulu mod\_auth\_openidc, kterou popisuji v sekci \hyperref[sec:mod-conf]{4.1.3}.

\subsection{Konfigurace nastavení Apache modulu mod\_auth\_openidc} 
\label{sec:mod-conf}
Předpokladem pro kompletní zprovoznění OpenID Connect modulu je spuštění příkazu \texttt{a2enmod auth\_openidc} který zapnul tento dříve nainstalovaný modul. Potom následovala konfigurace nainstalovaného modulu v souboru Apache serveru \texttt{/etc/apache2/sites-available/default-ssl}. Do zmíněného souboru byla přidána následující konfigurace.

\begin{itemize}
    \item \textbf{LoadModule auth\_openidc\_module /usr/lib/apache2\\ /modules/mod\_auth\_openidc.so} značí načtení požadovaného modulu, 
    \item \textbf{OIDCProviderMetadataURL https://login.cesnet.cz/oidc/.well-known/openid-configuration} představuje adresu odkazu na konfiguraci poskytovatele OpenID Connect, 
    \item \textbf{OIDCProviderMetadataRefreshInterval 3600} nastavení intervalu pro obnovení metadat poskytovatele OpenID Connect, 
    \item \textbf{OIDCClientID client\_id} nastavuje identifikaci klienta, která byla vygenerovaná v rámci registrace,
    \item \textbf{OIDCClientSecret client\_secret} definuje sdílené tajemství mezi klientem a poskytovatelem OpenID Connect, také se použije hodnota, která byla dříve vygenerovaná při registraci,
    \item \textbf{OIDCScope "openid email profile address phone organization eppns"} slouží pro nastavení \textbf{Scopes}, které jsou požadované od poskytovatele OpenID Connect,
    \item \textbf{OIDCRedirectURI /oauth2callback} značí adresu na kterou bude prohlížeč přesměrován po autentizaci, musí mít stejnou hodnotu jako zvolená hodnota při registraci klienta, 
    \item \textbf{OIDCCryptoPassphrase randompassword} představuje heslo pro šifrování cookie a cache dat. 
\end{itemize}

Po nastavení uvedené konfigurace, mohlo dojít k 

\section{Implementace autentizačního a autorizačního systému pro systém Adobe Connect}
\label{ACImpl}
\section{Implementace autentizačního a autorizačního systému pro systém Shongo}
\subsection{Implementace na straně autentizačního a autorizačního serveru}
\subsection{Implementace na straně systému Shongo}
\label{ShongoImpl}
\chapter{Závěr}
\printbibliography[title={Literatura}]
\end{document}
